\chapter{Floating point arithmetic}%
\label{cha:rounding_errors}
\minitoc

When we study numerical algorithms in the next chapters,
we assume implicitly that the operations involved are performed exactly.
On a computer, however, only a subset of the real numbers can be stored and,
consequently, many arithmetic operations are performed only approximately.
This is the source of the so-called \emph{round-off errors}.
Whether or not these inevitable errors are sufficiently small to be neglected in a numerical method
depends on the \emph{numerical stability} of the method.
A numerical method is said to be \emph{numerically unstable} is small errors are magnified,
and \emph{numerically stable} otherwise.
% these errors generally need not be a cause worry when they appear in methods that are numerically stables.



Our aims in this chapter are the following:
\begin{itemize}
    \item To describe the set of floating point numbers that can be represented in a typical computer;
    \item To briefly present how these numbers are usually encoded;
    \item To understand how arithmetic operations between floating point numbers work.
\end{itemize}

\section{An example: numerical differentiation}%
To illustrate this point,
consider the following program for calculating the derivative of the function $f(x) = \log(x)$ at $x = 1$.
\[
    f'(x) \approx \frac{\log(x + \varepsilon) - \log(x)}{\varepsilon}
\]
\begin{minted}{julia}
f(x) = log(x)
end
\end{minted}

\begin{exercise}
    [Avoiding overflow]
    Write a code to calculate the weighted average
    \[
        S := \frac
        {\sum_{x=0}^{L} \e^x i}
        {\sum_{x=0}^{L} \e^x}, 
        \qquad L = 1000.
    \]
\end{exercise}

% \begin{exercise}
%     Show that $31_{8} = 25_{10}$,
%     an equation sometimes written as~${\rm Oct} 31 = {\rm Dec} 25$.
% \end{exercise}

% \begin{example}
%     In the HTML language,
%     colors are represented by strings of the type $\texttt{\#rrggbb}$.
% \end{example}

% \begin{example}
%     [Computation of the standard deviation]
% \end{example}

% \begin{example}
%     [Calculation of the derivative]
%     A classic π example of cancellation is when calculating derivatives.
%     Consider the following approximation of $\frac{\d}{\d x}\log(x) \vert_{x=1}$:
%     \[
%         	\frac{\log(x + \varepsilon) - \log(x)}{\varepsilon}
%     \]
% \end{example}

% \begin{example}
%     [Second-degree equation]
%     Consider the equation
%     \[
%         x^2 - \varepsilon x + 1 = 0
%     \]
% \end{example}

% \begin{theorem}
%     {Title of theorem}
%     \label{thm:test}
%     hello
% \end{theorem}

% \begin{example}
%     [A test example]
%     hello
% \end{example}

% \begin{lemma}
%     [Title of the lemma]
%     {Title of theorem}
%     \label{lemma:test}
%     hello
% \end{lemma}

% \begin{remark}
%     [Hello]
%     test
% \end{remark}

% \begin{theorem}
%     {Title of theorem}
%     test
% \end{theorem}

% \cref{lemma:test}

% \section{Discussion and bibliography}%
% \label{sec:discussion_and_bibliograhpy}
% The IEEE754 standard is defined in~\cite{ieee754}
% % In this chapters,
% % we covered the fundamentals of floating point arithmetics,
