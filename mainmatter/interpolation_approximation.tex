\chapter{Interpolation and approximation}%
\label{cha:interpolation_and_approximation}

\minitoc

\section*{Introduction}
In this chapter,
we study numerical methods for interpolating and approximating functions.
The Cambridge dictionary defines interpolation as \emph{the addition of something different in the middle of a text, piece of music, etc.~or the thing that is added}.
The concept of interpolation in mathematics is consistent with this definition;
interpolation consists in finding, given a set of points~$(x_i, y_i)$,
a function~$f$ in a finite-dimensional space that goes through these points.
Throughout this course, you have used the~\julia{plot} function in Julia,
which performs piecewise linear interpolation,
but there are a number of other standard interpolation methods.
Our first goal in this chapter is to present an overview of these methods and the associated error estimates.

In the second part of this chapter,
we focus of \emph{function approximation},
which is closely related to the subject of mathematical interpolation.
Indeed, a simple manner for approximating a general function by another one in a finite-dimensional space is to select a set of real numbers on the $x$ axis,
called \emph{nodes}, and find the associated interpolant.
As we shall demonstrate, not all sets of interpolation nodes are equal,
and special care is required in order to avoid undesired oscillations.
The field of function approximation is vast,
so our aim in this chapter is to present only an introduction to the subject.
In order to quantify the quality of an approximation,
a metric on the space of functions,
or a subset thereof, must be specified in order to measure errors.
Without a metric, saying that two functions are close is almost meaningless!
% Consider, for example, the function $f\colon [0, 1] \to \real; x \mapsto 0$ and the approximations $\widehat f_1(x) = x^{100}$ and $\widehat f_2(x) = 0.01$

\section{Function interpolation}
Assume that we are given $n+1$ nodes $x_1, \dotsc, x_n$ on the $x$ axis,
together with the values $u_0, \dotsc, u_n$ taken by an unknown function~$u(x)$ when evaluated at these points,
suppose that we are looking for an interpolation~$\widehat u(x)$ in a subspace~$\Span \{\varphi_0, \dotsc, \varphi_n\}$
of the space of continuous functions, i.e.~an interpolating function of the form
\[
    \widehat u(x) = \alpha_0 \varphi_0(x) + \dotsb + \alpha_n \varphi_n(x),
\]
where $\alpha_0, \dotsc, \alpha_n$ are real coefficients.
In order for~$\widehat u(x)$ to be an interpolating function,
we must require that
\[
    \forall i \in \{0, \dotsc, n\}, \qquad
    \widehat u(x_i) = u_i.
\]
This leads to a linear system of $n+1$ equations and $n+1$ unknowns,
the latter being the coefficients~$\alpha_0, \dotsc, \alpha_n$.
This system of equations in matrix form reads
\begin{equation}
    \label{eq:linear_system_interpolation}
    \begin{pmatrix}
        \varphi_0(x_0) & \varphi_1(x_0) & \hdots & \varphi_n(x_0) \\
        \varphi_0(x_1) & \varphi_1(x_1) & \hdots & \varphi_n(x_1) \\
        \vdots & \vdots & & \vdots \\
        \varphi_0(x_n) & \varphi_1(x_n) & \hdots & \varphi_n(x_n)
    \end{pmatrix}
    \begin{pmatrix}
        \alpha_0 \\
        \alpha_1 \\
        \vdots \\
        \alpha_n
    \end{pmatrix}
    =
    \begin{pmatrix}
        u_0 \\
        u_1 \\
        \vdots \\
        u_n
    \end{pmatrix}.
\end{equation}

\subsection{Vandermonde matrix}
Since polynomials are very convenient for evaluation, integration, and differentiation,
they are a natural choice for interpolation purposes.
The simplest basis of the subspace of polynomials of degree less than or equal to $n$ is given by the monomials:
\[
    \varphi_0(x) = 1,
    \qquad
    \varphi_1(x) = x,
    \qquad \dotsc, \qquad
    \varphi_n(x) = x^n.
\]
In this case,
the linear system~\eqref{eq:linear_system_interpolation} for determining the coefficient of the interpolant reads
\begin{equation}
    \label{eq:linear_system_interpolation_poly}
    \begin{pmatrix}
        1 & x_0 & \hdots & x_0^n \\
        1 & x_1 & \hdots & x_1^n \\
        \vdots & \vdots & & \vdots \\
        1 & x_n & \hdots & x_n^n
    \end{pmatrix}
    \begin{pmatrix}
        \alpha_0 \\
        \alpha_1 \\
        \vdots \\
        \alpha_n
    \end{pmatrix}
    =
    \begin{pmatrix}
        u_0 \\
        u_1 \\
        \vdots \\
        u_n
    \end{pmatrix}.
\end{equation}
The matrix on the left-hand side is called a \emph{Vandermonde} matrix.
If the abcissae $x_0, \dotsc, x_n$ are distinct,
then this is a full rank matrix,
and so~\eqref{eq:linear_system_interpolation_poly} admits a unique solution,
implying as a corollary that the interpolating polynomial exists and is unique.
It is possible to show that the condition number of the Vandermonde increases dramatically with $n$,
and so solving~\eqref{eq:linear_system_interpolation_poly} is not a viable method in practice for calculating the interpolating polynomial.

\subsection{Lagrange interpolation formula}
One may wonder whether polynomial basis functions $\varphi_0, \dotsc, \varphi_n$ can be defined in such a manner that
the matrix in~\eqref{eq:linear_system_interpolation} is the identity matrix.
The answer to this question is positive;
it suffices to take as a basis the \emph{Lagrange polynomials},
which are given by
\[
    \varphi_{i}(x)
    = \frac{(x - x_0) (x - x_1) \dotsc (x - x_{i-1}) (x - x_{i+1}) \dotsc (x - x_n)}
    {(x_i - x_0) (x_i - x_1) \dotsc (x_i - x_{i-1}) (x_i - x_{i+1}) \dotsc (x_i - x_n)}.
\]
It is simple to check that
\[
    \varphi_i(x_j) =
    \delta_{i,j} =
    \begin{cases}
        1 & \text{if $i = j$}, \\
        0 & \text{otherwise.}
    \end{cases}
\]
Finding the interpolant in this basis is immediate:
\[
    \widehat u(x) = u_1 \varphi_1(x) + \dotsb + u_n \varphi_n(x).
\]
While simple, this approach to polynomial interpolation has a couple of disadvantages:
first, evaluating $\widehat u(x)$ is computationally costly when $n$ is large and,
second, all the basis functions change when adding new interpolation nodes.

\subsection{Newton interpolation}
By Taylor's formula,
any polynomial~$p$ of degree $n$ may be expressed as
\begin{equation}
    \label{eq:taylor}
    p(x) = p(0) + p'(0) x + \frac{p''(0)}{2} x^2 + \dotsc + \frac{p^{(n)}(0)}{n!} x^n.
\end{equation}
In other words, the constant coefficient can be obtained by evaluating the polynomial at 0,
the linear coefficient can be identified by evaluating the first derivative at 0,
and so on.
Assume now that we are given the values taken by~$p$ when evaluated at the integer numbers $\{0, \dotsc, n\}$,
which we denote by $p_0, \dotsc, p_n$.
We ask the following question,
can we find a formula similar to~\eqref{eq:taylor} where,
instead of the derivatives, we have approximations of the derivatives?
To this end, we introduce the difference operator $\Delta$,
which is given by
\[
    \Delta: (p_0, p_1, \dotsc) \mapsto (p_1 - p_0, p_2 - p_1, \dotsc).
\]
The operator~$\Delta$ is a linear operator on the space of sequences.
Like the differentiation operator,
it maps the constant sequence to 0,
and the linear sequence $(0, 1, 2, \dotsc)$ to the constant sequence~$(1, 1, 1, \dotsc)$.

\begin{exercise}
    using the Gregory--Newton formula,
    Find an expression for
    \[
        \sum_{i=1}^{n} n^4.
    \]
\end{exercise}

\begin{exercise}
    Using the Gregory--Newton formula,
    show that
    \[
        \forall n \in \nat,
        \qquad 2^n = 1 + n + \frac{n(n-1)}{2!} + \frac{n(n-1)(n-2)}{3!} + \frac{n(n-1)(n-2)(n-3)}{4!} + \dotsb
    \]
\end{exercise}
