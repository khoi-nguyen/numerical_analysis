\chapter{Numerical integration}
\label{cha:quadrature}

\minitoc

\section*{Introduction}
Integrals are ubiquitous in science and mathematics.
In this chapter,
we are concerned with the problem of calculating numerically integrals of the form
\begin{equation}
    \label{eq:integral}
    I = \int_{\Omega} u(\vect x) \, \d \vect x,
\end{equation}
Perhaps somewhat surprisingly,
the numerical calculation of such integrals when $n \gg 1$ is still a very active area of research today.
In this chapter,
we will focus for simplicity on the one-dimensional setting where $\Omega = [a, b] \subset \real$.
We assume throughout this chapter that the function~$u$ is Riemann-integrable.
This means that
\[
    I = \lim_{h \to 0}
    \sum_{i=0}^{n-1} u(t_i) (z_{i+1} - z_i),
\]
where $a = z_0 < \dotsb < z_n = b$ is a partition of the interval~$[a, b]$ such that the maximum spacing between successive~$x$ values is equal to~$h$,
and with $t_i \in [x_i, x_{i+1}]$ for all $i \in \{0, \dotsc, n-1\}$.

All the numerical integration formulas that we present in this chapter are based on a deterministic approximation of the form
\begin{equation}
    \label{eq:deterministic_integration}
    \widehat I = \sum_{i=0}^{m} w_i u(x_i),
\end{equation}
where~$x_0 < \dotsc < x_n$ are the \emph{integration nodes} and $w_0, \dotsc, w_n$ are the \emph{integration weights}.
In many cases, integration formulas contain a small parameter that can be changed to improve the accuracy of the approximation.
In methods based on equidistant interpolation nodes,
for example, this parameter encodes the distance between nodes and is typically denoted by~$h$,
and we often use the notation~$\widehat I_h$ to emphasize the dependence of the approximation on~$h$.
The difference~$E_h = I - I_h$ is called the \emph{integration error} or \emph{discretization error},
and the \emph{degree of precision} of an integration method is the smallest integer number~$d$ such that
the integration error is zero for all the polynomials of degree less than or equal to~$d$.

We observe that,
without loss of generality,
we can consider that the integration interval is equal to~$[-1, 1]$.
Indeed, using the change of variable
\begin{align}
    \notag
    \zeta\colon &[-1, 1] \to [a, b]; \\
    \label{eq:change_of_variable_integration}
                &y \mapsto \frac{b+a}{2} + \frac{(b-a)}{2} y,
\end{align}
we have
\begin{equation}
    \label{eq:change_of_variable_equivalence}
    \int_{a}^{b} u(x) \, \d x
    = \int_{-1}^{1} u\bigl(\zeta(y)\bigr) \, \zeta'(y) \, \d y
    = \frac{b-a}{2}\int_{-1}^{1} u \circ \zeta (y) \, \d y,
\end{equation}
and the right-hand side is the integral of $u \circ \zeta$ over the interval~$[-1, 1]$.

\section{The Newton--Cotes method}
\label{sec:newton_cotes}

Given a set of equidistant points $-1 = x_0 < \dotsb < x_m = 1$,
a natural method for approximating the integral~\eqref{eq:integral} of a function~$u\colon [-1, 1] \to \real$
is to first construct the interpolating polynomial~$\widehat u$ through the nodes,
and then calculate the integral of this polynomial.
By construction, this method is exact for polynomials of degree up to~$m$,
and so the degree of precision is equal to \emph{at least}~$m$.
Let $\varphi_0, \dotsc, \varphi_m$ denote the Lagrange polynomials associated with the integration nodes.
Then we have
\[
    I \approx \int_{-1}^{1} \widehat u(x) \, \d x
    = \int_{-1}^{1} \sum_{i=0}^{n} u(x_i) \varphi_i u(x) \, \d x
    = \sum_{i=0}^{n} u(x_i) \underbrace{\int_{-1}^{1}  \varphi_i u(x)  \, \d x}_{w_i}.
\]
The weights are independent of the function~$u$,
and so they can be calculated a priori.
The class of integration methods obtained using this approach are known as \emph{Newton--Cotes methods}.
We present a few particular cases:
\begin{itemize}
    \item
        $m = 1$, $d = 1$ (trapezoidal rule):
        \begin{equation}
            \label{eq:trapezoidal_rule}
            \int_{-1}^{1} u(x) \, dx
            \approx u(-1) + u(1).
        \end{equation}

    \item
        $m = 2$, $d = 3$ (Simpson's rule):
        \begin{equation}
            \label{eq:simpsons}
            \int_{-1}^{1} u(x) \, dx
            \approx \frac{1}{3} u(-1) + \frac{4}{3} u(0) + \frac{1}{3} u(1).
        \end{equation}

    \item
        $m = 3$, $d = 3$ (Simpson's $\frac{3}{8}$ rule):
        \[
            \int_{-1}^{1} u(x) \, dx
            \approx \frac{1}{4} u(-1) + \frac{3}{4} u(-1/3) + \frac{3}{4} u(1/3) + \frac{1}{4} u(1).
        \]

    \item
        $m = 4$, $d = 5$ (Bode's rule):
        \[
            \int_{-1}^{1} u(x) \, dx
            \approx \frac{7}{45} u(-1) + \frac{32}{45} u\left(-\frac{1}{2}\right) + \frac{12}{45} u\left(0\right) + \frac{32}{45} u\left(\frac{1}{2}\right) + \frac{7}{45} u(1).
        \]
\end{itemize}
In principle,
this approach could be employed in order to construct integration rules of arbitrary high degree of precision.
In practice, however, the weights become more and more imbalanced as the number of interpolation points increases,
with some of them becoming negative.
As a result, roundoff errors become increasingly detrimental to accuracy as the degree of precision increases.
In addition, in cases where the interpolating polynomial does not converge to~$u$,
for example if~$u$ is Runge's function,
the approximate integral may not even converge to the correct value in the limit as~$m \to \infty$ in exact arithmetic!

Note that,
although it is based on a quadratic polynomial interpolation,
Simpson's rule~\eqref{eq:simpsons} has a degree of precision equal to~$3$.
This is because any integration rule with symmetric nodes and weights on either side of $x=0$ is exact for odd functions,
in particular $x^3$.
Likewise, the degree of precision of Bode's rule is equal to 5.

\section{Composite methods based on piecewise interpolation}
A natural alternative to the approach presented in~\cref{sec:newton_cotes}
is to construct an integration rule using piecewise polynomial interpolation,
which we studied in~\cref{sub:piecewise_interpolation}.
After partitioning the integration interval in a number of subintervals,
the integral can be approximated by using one of the rules presented in~\eqref{sec:newton_cotes} within each subinterval.

Let us illustrate the composite approach with an example.
To this end,
we introduce a partition $a = x_0 < \dotsb < x_m = b$ of the interval $[a, b]$ and
assume that the nodes are equidistant with $x_{i+1} - x_i = h$.
Using~\eqref{eq:change_of_variable_equivalence},
we first generalize~\eqref{eq:trapezoidal_rule} to an interval~$[x_i, x_{i+1}]$ as follows:
\[
    \int_{x_i}^{x_{i+1}} u(x) \, \d x
    = \int_{-1}^{1} u \circ \zeta (y) \, \d y
    \approx u \circ \zeta(-1)  + u \circ \zeta(1)
    = \frac{h}{2} \bigl(u(x_i) + u(x_{i+1})\bigr),
\]
where $\approx$ in this equation indicates approximation using the trapezoidal rule.
Applying this approximation to each subinterval of the partition,
we obtain the composite trapezoidal rule:
\begin{align}
    \notag
    \int_{a}^{b} u(x) \, \d x
    &= \sum_{i=0}^{n-1} \int_{x_i}^{x_{i+1}} u(x) \, \d x
    \approx
    \frac{h}{2}\sum_{i=0}^{n-1} \bigl( u(x_i) + u(x_{i+1}) \bigr) \\
    \label{eq:composite_trapezoidal_rule}
    &= h \bigl( u(x_0) + 2 u(x_1) + 2 u(x_2) + \dotsb + 2 u(x_{n-2}) + 2 u(x_{n-1}) + u(x_n) \bigr).
\end{align}
Like the trapezoidal rule~\eqref{eq:trapezoidal_rule},
the composite trapezoidal rule~\eqref{eq:composite_trapezoidal_rule} has a degree of precision equal to 1.
However,
the accuracy of the method depends on the parameter~$h$,
which represents the width of each subinterval:
for very small~$h$,
equation~\eqref{eq:composite_trapezoidal_rule} is expected to provide a good approximation of the integral.
An error estimate can be obtained directly from the upper bound~\eqref{eq:error_estimate_piecewise_interpolation} on the piecewise interpolation error,
provided that we assume that~$u \in C^2([a, b])$.
Denoting by $\widehat I_h$ the approximate integral calculated using~\eqref{eq:composite_trapezoidal_rule},
and by~$\widehat u_h$ the piecewise linear interpolation of~$u$,
we have
\[
    \abs{I - \widehat I_h}
    = \abs*{\int_{a}^{b} u(x) - \widehat u_h(x) \, \d x}
    \leq \int_{a}^{b} \abs{u(x) - \widehat u_h(h)} \, \d x
    \leq \frac{b-a}{8} \left(  \sup_{\xi \in [a, b]} \abs{u''(\xi)} \right) h^2.
\]
The integration error therefore scales as~$\mathcal O(h^2)$.
(Strictly speaking, we have shown only that the integration error admits an upper bound that scales at $\mathcal O(h^2)$,
but it turns out that the dependence on~$h$ of this bound is optimal).

\section{Exercises}
\begin{exercise}
    Derive the Simpson's integration rule~\eqref{eq:simpsons}.
\end{exercise}

\begin{exercise}
    Derive the composite Simpson integration rule.
\end{exercise}

\begin{exercise}
    Show that the integration error for the composite Simpson rule satisfies the following bound:
    \[
        \abs{I - \widehat I_h}
        \leq C (b-a) \left( \sup_{\xi \in [a, b]} \abs*{u^{(4)}(\xi)} \right) h^4,
    \]
    for an appropriate constant~$C$.
\end{exercise}

\section{Discussion and bibliography}
In this chapter,
we covered mainly \emph{deterministic} integration formulas.
The presentation of part of the material follows that in~\cite{Legat}.
Much of the research around the calculation of high-dimensional integrals today is concerned with~\emph{probabilistic} integration methods using Monte Carlo approaches.
These methods are based on the connection between integrals and expectations.
For example, the integral
\[
    I = \int_{0}^{1} x^2 \, \d x
\]
may be expressed as the expectation $\expect [X^2]$,
where~$\expect$ is the expectation operator and~$X \sim \mathcal U(0, 1)$ is a uniformly distributed random variable over the interval~$[0, 1]$.
Therefore, in practice, $I$ may be approximated by generating a large number of samples $X_1, X_2,\dotsc$ from the distribution~$\mathcal U(0, 1)$
and averaging $f(X_i)$ over all these samples.
\begin{minted}{julia}
    n = 1000
    f(x) = x^2
    X = rand(n)
    I = (1/n) * sum(f.(X))
\end{minted}
The main advantage of this approach is that it generalizes very easily to high-dimensional and infinite-dimensional settings.
