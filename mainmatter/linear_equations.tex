\chapter{Solution of linear systems of equation}
\label{cha:solution_of_linear_systems}

Systems of linear equations appear in a wide variety of applications.
They naturally arise after discretization of linear partial differential equation,
which are ubiquitous in physics.
Laplace and Poisson equation is  Newtonian gravity, heat transfer, fluid flows.

% Whether or not these inevitable errors are sufficiently small to be neglected in a numerical method
% depends in general on the \emph{numerical stability} of the method.
% A numerical method is said to be \emph{numerically unstable} is small errors are magnified,
% and \emph{numerically stable} otherwise.
% these errors generally need not be a cause worry when they appear in methods that are numerically stables.

\begin{itemize}
    \item 
        In \cref{sec:conditioning},
        we introduce the concept of \emph{conditioning}.
        The condition number associated with a problem provides information on the sensitivity of the solution to errors in the data,
        so it is useful for determining the potential impact of round-off errors.
\end{itemize}

\section{Conditioning}%
\label{sec:conditioning}
