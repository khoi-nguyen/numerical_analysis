\chapter{Solution of linear systems of equation}
\label{cha:solution_of_linear_systems}
This chapter is devoted to the numerical solution of linear problems of the following form:
\begin{equation}
    \label{eq:linear_system}
    \text{Find $\vect x \in \real^n$ such that} \qquad
    \mat A \vect x = \vect b,
    \qquad \mat A \in \real^{n \times n},
    \qquad \vect b \in \real^n.
\end{equation}
Systems of this type appear in a variety of applications.
They naturally arise in the context of linear partial differential equations,
which we use as main motivating example.
Partial differential equations govern a wide range of physical phenomena including heat propagation, gravity, and electromagnetism,
to mention just a few.
Linear systems in this context often have a particular structure:
the matrix $\mat A$ is generally very sparse,
which means that most of the entries are equal to 0,
and it is often symmetric and positive definite,
provided that these properties are satisfied by the underlying operator.

There are two main approaches for solving linear systems:
\begin{itemize}
    \item
        Direct methods enable to calculate the exact solution to systems of linear equations,
        up to round-off errors.
        Although this is an attractive property,
        they are usually too computationally costly for large systems:
        The cost of inverting a general $n \times n$ matrix,
        measured in number of floating operations,
        scales as $n^3$!

    \item
        Iterative methods, on the other hand,
        enable to progressively calculate increasingly accurate approximations of the solution.
        Iterations may be stopped once the \emph{the residue} is sufficiently small.
        These methods are often preferable when the dimension $n$ of the linear system is very large.
\end{itemize}

This section is organized as follows.
\begin{itemize}
    \item
        In \cref{sec:conditioning},
        we introduce the concept of \emph{conditioning}.
        The condition number of a matrix provides information on the sensitivity of the solution to perturbations of the right-hand side $\vect b$ or matrix $\mat A$.
        It is useful, for example, in order to determine the potential impact of round-off errors.
\end{itemize}

\section{Conditioning}%
\label{sec:conditioning}

The condition number for a given problem measures the sensitivity of the solution to some input data.
In order to define this concept precisely,
we consider a general problem of the form~$F(x, d) = 0$,
with unknown $x$ and data $d$.
The linear system~\eqref{eq:linear_system} can be recast in this form,
with the input data equal to $\vect b$ or $\mat A$ or both.
We denote the solution corresponding to a perturbed input data $d + \Delta d$ by $x + \Delta x$.
For the general problem,
we define the absolute and relative condition numbers as follows.

\begin{definition}
    [Condition number for abstract problem]
    The absolute and relative condition numbers with respect to perturbations of $d$ are defined as
    \[
        K_{\rm abs}(d) = \lim_{\varepsilon \to 0} \left( \sup_{\norm{\Delta d} \leq \varepsilon} \frac{\norm{\Delta x}}{\norm{\Delta d}} \right),
        \qquad
        K(d) = \lim_{\varepsilon \to 0} \left( \sup_{\norm{\Delta d} \leq \varepsilon} \frac{\norm{\Delta x} / \norm{x}}{\norm{\Delta d} / \norm{d}} \right).
    \]
    The short notation $K$ is reserved for the relative condition number,
    which is often more useful in applications.
\end{definition}

In the rest of this section,
we obtain an upper bound on the relative condition number of the linear system~\eqref{eq:linear_system} with respect to perturbations first of $\vect b$,
and then of $\mat A$.
We use the notation~$\norm{\placeholder}$ to denote both a vector norm on $\real^n$ and the induced operator norm on matrices.

\begin{proposition}
    [Perturbation of the right-hand side]
    \label{proposition:linear_perturbation_rhs}
    Let $\vect x + \Delta \vect x$ denote the solution to the perturbed equation $\mat A (\vect x + \Delta \vect x) = \vect b + \Delta \vect b$.
    Then it holds
    \begin{equation}
        \label{eq:linear_perturbation_rhs}
        \frac{\norm{\Delta \vect x}}{\norm{\vect x}} \leq \norm{\mat A} \norm{\mat A^{-1}} \, \frac{\norm{\Delta \vect b}}{\norm{\vect b}},
    \end{equation}
\end{proposition}
\begin{proof}
    It holds by definition that $\mat A \Delta \vect x = \Delta \vect b$.
    Therefore, we have
    \begin{equation}
        \label{eq:linear_perturbation_rhs_to_rearrange}
        \norm{\Delta \vect x}
        = \norm{\mat A^{-1} \Delta \vect b}
        \leq \norm{\mat A^{-1}} \norm{\Delta \vect b}
        = \frac{\norm{\mat A \vect x}}{\norm{\vect b}} \norm{\mat A^{-1}} \norm{\Delta \vect b}
        \leq \frac{\norm{\mat A} \norm{\vect x}}{\norm{\vect b}} \norm{\mat A^{-1}} \norm{\Delta \vect b}.
    \end{equation}
    Here we employed~\eqref{eq:submultiplicative_mat_vec},
    proved in \cref{cha:vectors_and_matrices},
    in the first and last inequalities.
    Rearranging the inequality~\eqref{eq:linear_perturbation_rhs_to_rearrange},
    we obtain~\eqref{eq:linear_perturbation_rhs}.
\end{proof}
\Cref{proposition:linear_perturbation_rhs} shows that
the relative condition number of~\eqref{eq:linear_system} with respect to perturbations of the right-hand side is bounded from above by $\norm{\mat A} \norm{\mat A^{-1}}$.
\Cref{exercise:linear_sharp_inequality} shows that there are values of $\vect x$ and $\Delta \vect b$ for which the inequality~\eqref{eq:linear_perturbation_rhs} is sharp.

Studying the impact of perturbations of the matrix~$\mat A$ is slightly more difficult,
because this time the variation of the solution~$\Delta \vect x$ does not depend linearly on the perturbation.
\begin{proposition}
    [Perturbation of the matrix]
    \label{proposition:linear_perturbation_matrix}
    Let $\vect x + \Delta \vect x$ denote the solution to the perturbed equation $(\mat A + \Delta \mat A) (\vect x + \Delta \vect x) = \vect b$.
    If $\mat A$ is invertible and $\norm{\Delta \mat A} < \norm{\mat A^{-1}}^{-1}$,
    then
    \begin{equation}
        \label{eq:linear_perturbation_matrix}
        \frac{\norm{\Delta \vect x}}{\norm{\vect x}}
        \leq \norm{\mat A} \norm{\mat A^{-1}} \frac{\norm{\Delta \mat A}}{\norm{\mat A}}
        \left(\frac{1}{1 - \norm{\mat A^{-1} \Delta \mat A}} \right).
    \end{equation}
\end{proposition}
Before proving this result,
we show the following ancillary lemma.
\begin{lemma}
    \label{lemma:linear_inverse_neumann}
    Let $\mat B \in \real^{n \times n}$ be such that $\norm{\mat B} \leq 1$.
    Then $\mat I - \mat B$ is invertible and
    \begin{equation}
        \label{eq:linear_bound_inverse_perturbation_identity}
        \norm{(\mat I - \mat B)^{-1}}
        \leq \frac{1}{1 - \norm{\mat B}},
    \end{equation}
    where $\mat I \in \real^{n \times n}$ is the identity matrix.
\end{lemma}
\begin{proof}
    It holds for any matrix $\mat B \in \real^{n \times n}$ that
    \[
        \mat I - \mat B^{n+1} = (\mat I - \mat B)(\mat I + \mat B + \dotsb + \mat B^n)
    \]
    Since $\norm{\mat B} < 1$ in a submultiplicative matrix norm,
    both sides of the equation are convergent in the limit as $n \to \infty$,
    with the left-hand side converging to identity matrix $\mat I$.
    Equating the limits,
    we obtain
    \[
        \mat I = (\mat I - \mat B) \sum_{i=0}^{\infty} \mat B^i.
    \]
    This implies that $(\mat I - \mat B)$ is invertible with inverse
    given by the so-called \emph{Neumann} series
    \begin{equation*}
        (\mat I - \mat B)^{-1} = \sum_{i=0}^{\infty} \mat B^i.
    \end{equation*}
    Applying the triangle inequality repeatedly,
    and then using the submultiplicative property of the norm,
    we obtain
    \[
        \forall n \in \nat,
        \qquad
        \norm*{\sum_{i=0}^{n} \mat B^i}
        \leq \sum_{i=0}^{n} \norm{\mat B^i}
        \leq \sum_{i=0}^{n} \norm{\mat B}^i
        = \frac{1}{1 - \norm{\mat B}}.
    \]
    where we used the summation formula for geometric series in the last equality.
    Letting $n \to \infty$ in this equation and
    using the continuity of the norm enables to conclude the proof.
\end{proof}

\begin{proof}
    [Proof of \cref{proposition:linear_perturbation_matrix}]
    Left-multiplying both side by $\mat A^{-1}$,
    we obtain
    \begin{equation}
        \label{eq:linear_perturbation_matrix_initial}
        (\mat I + \mat A^{-1} \Delta \mat A) (\vect x + \Delta \vect x) = \vect x
        \quad \Leftrightarrow \quad
        (\mat I + \mat A^{-1} \Delta \mat A) \Delta \vect x = - \mat A^{-1} \Delta \mat A \vect x.
    \end{equation}
    Since $\norm{\mat A^{-1} \Delta \mat A} \leq \norm{\mat A^{-1}} \norm{\Delta \mat A} < 1$ by assumption,
    we deduce~\cref{lemma:linear_inverse_neumann} that the matrix on the left-hand side is invertible
    with a norm bounded as in~\eqref{eq:linear_bound_inverse_perturbation_identity}.
    Consequently,
    using in addition the assumed submultiplicative property of the norm,
    we deduce that
    \[
        \norm{\Delta \vect x}
        = \norm{(\mat I + \mat A^{-1} \Delta \mat A)^{-1} \mat A^{-1} \Delta \mat A \vect x}
        \leq \frac{\norm{\mat A^{-1} \Delta \mat A}}{1 - \norm{\mat A^{-1} \Delta \mat A}} \norm{\vect x}.
    \]
    which enables to conclude the proof.
\end{proof}
Using \cref{proposition:linear_perturbation_matrix},
we deduce that the relative condition number of~\eqref{eq:linear_system} with respect to perturbations of the matrix $\mat A$ is also bounded from above by $\norm{\mat A} \norm{\mat A^{-1}}$,
because the term between brackets on the right-hand side of~\eqref{eq:linear_perturbation_matrix} converges to 1 as $\norm{\Delta \mat A} \to 0$.


\Cref{proposition:linear_perturbation_rhs,proposition:linear_perturbation_matrix} show that
the condition number, with respect to perturbations of either~$\vect b$ or~$\mat A$,
depends only on $\mat A$.
This motivates the following definition.
\begin{definition}
    [Condition number of a matrix]
    The condition number of a matrix $\mat A$ associated to a vector norm $\norm{\placeholder}$ is defined as
    \[
        \kappa(\mat A) = \norm{\mat A} \norm{\mat A^{-1}}.
    \]
    The condition number for the vector $p$-norm,
    defined in \cref{definition:pnorm_vector},
    is denoted by $\kappa_p(\mat A)$.
\end{definition}
Since the 2-norm of an invertible matrix $\mat A \in \real^{n \times n}$ coincides with the spectral radius $\rho(\mat A^\t \mat A)$,
the condition number $\kappa_2$ corresponding to the $2$-norm is equal to
\[
    \kappa_2(\mat A) = \frac{\lambda_{\max}(\mat A^\t \mat A)}{\lambda_{\min}(\mat A^\t \mat A)},
\]
where $\lambda_{\max}$ and $\lambda_{\min}$ are the maximal and minimal (both real and positive) eigenvalues of $\mat A$.
\begin{example}
    [Perturbation of the matrix]
    Consider the following linear system
    with perturbed matrix
    \[
        (\mat A + \Delta \mat A)
        \begin{pmatrix}
            x_1 \\
            x_2
        \end{pmatrix}
        = \begin{pmatrix}
            0 \\
            .01
        \end{pmatrix},
        \qquad
        \mat A
        = \begin{pmatrix}
            1 & 0 \\
            0 & .01
        \end{pmatrix},
        \qquad
        \Delta \mat A =
        \begin{pmatrix}
            0 & 0 \\
            0 & \varepsilon
        \end{pmatrix},
    \]
    where $0 < \varepsilon \ll .01$.
    Here the eigenvalues of $\mat A$ are given by $\lambda_1 = 1$ and $\lambda_2 = 0.01$.
    The solution when $\varepsilon = 0$ is given by $(0, 1)^\t$,
    and the solution to the perturbed equation is
    \[
        \begin{pmatrix}
        x_1 + \Delta x_1 \\
        x_2 + \Delta x_2
        \end{pmatrix}
        =
        \begin{pmatrix}
            0 \\
            \frac{1}{1 + 100 \varepsilon}
        \end{pmatrix}.
    \]
    Consequently, we deduce that
    \[
        \frac{\norm{\Delta \vect x}}{\norm{\vect x}}
        = \abs*{\frac{100 \varepsilon}{1 + 100 \varepsilon}}
        \approx 100 \varepsilon
        = 100 \frac{\norm{\Delta \mat A}}{\norm{\mat A}}.
    \]
    In this case,
    the relative impact of perturbations of the matrix is close to $\kappa_2(\mat A) = 100$.
\end{example}

\begin{exercise}
    \label{exercise:linear_sharp_inequality}
    In the simple case where $\mat A$ is symmetric,
    find values of $\vect x$, $\vect b$ and $\Delta \vect b$ for which the inequality~\eqref{eq:linear_perturbation_rhs} is in fact an equality?
\end{exercise}

\section{Direct solution method}%
\label{sec:direct_solution_method}
In this section,
we present the \emph{direct method} for solving linear systems of the form~\eqref{eq:linear_system}
with a general invertible matrix~$\mat A \in \real^{n \times n}$.
The direct method can be decomposed into three steps:
\begin{itemize}
    \item
        First calculate the so-called $\mat L \mat U$ decomposition of $\mat A$,
        i.e.\ find an upper triangular matrix~$\mat U$ and a lower triangular matrix~$\mat L$ such that
        \(
            \mat A = \mat L \mat U.
        \)

    \item
        Then solve
        \(
            \mat L \vect y = \vect b.
        \)
         using a method called \emph{forward substitution}.

    \item
        Finally, solve
        \(
            \mat U \vect x = \vect y.
        \)
         using a method called \emph{backward substitution}.
         By construction, the solution $\vect x$ thus obtained is a solution to~\eqref{eq:linear_system}.
\end{itemize}

\subsection{LU decomposition}%
\label{sub:lu_decomposition}

We begin by introducing the concept of \emph{Gaussian transformation}.
\begin{definition}
    A Gaussian transformation is a matrix of the form $\mat M_k = \mat I - \vect c^{(k)} \vect e_k$,
    where~$\vect e_k$ is the column vector with entry at index $k$ equal to 1 and all the other entries equal to zero,
    and $\vect c^{(k)}$ is a column vector of the following form:
    \[
        \vect c^{(k)} =
        \begin{pmatrix}
            0 & 0 & \dots & 0 & c^{(k)}_{k+1} & c^{(k)}_{k+2} & \dots & c^{(k)}_n
        \end{pmatrix}^\t.
    \]
\end{definition}

The action of a Gaussian transformation left-multiplying a matrix $\mat A \in \real^{n \times n}$ is
to replace the rows from index $k + 1$ to index $n$ by a linear combination involving themselves and the $k$-th row.
To see this, let us denote by $(\vect r^{(i)})_{1 \leq i \leq n}$ the lines of a matrix $\mat T \in \real^{n \times n}$.
Then, we have
\[
    \mat M_k \mat T
    = \bigl(\mat I - \vect c^{(k)} \vect e_k\bigr) \mat T
    =
    \begin{pmatrix}
        1   \\
      & 1  \\
         & &  \ddots \\
        & & & 1 & & & \\
        & & & - c^{(k)}_{k+1} & 1  \\
        & & & \vdots & & \ddots \\
        & & & - c^{(k)}_n & & & 1 \\
    \end{pmatrix}
    \begin{pmatrix}
        \vect r^{(1)} \\
        \vect r^{(2)} \\
        \vdots \\
        \vect r^{(k)} \\
        \vect r^{(k+1)} \\
        \vdots \\
        \vect r^{(n)}
    \end{pmatrix}
    =
    \begin{pmatrix}
        \vect r^{(1)} \\
        \vect r^{(2)} \\
        \vdots \\
        \vect r^{(k)} \\
        \vect r^{(k+1)} - c^{(k)}_{k+1} \vect r^{(k)} \\
        \vdots \\
        \vect r^{(n)} - c^{(k)}_{n} \vect r^{(k)}
    \end{pmatrix}
\]
We show in \cref{exercise:inverse_gaussian_transformation} that
the inverse of a Gaussian transformation matrix is given by
\begin{equation}
    \label{eq:inverse_gaussian_transformation}
    (\mat I - \vect c^{(k)} \vect e_k)^{-1} = \mat I + \vect c^{(k)} \vect e_k.
\end{equation}
The idea of the $\mat L \mat U$ decomposition algorithm is to successively left-multiply $\mat A$
by Gaussian transformation matrices $\mat M_1$, then $\mat M_2$, etc.\
appropriately chosen in such a way that the matrix~$\mat A^{(k)}$,
obtained after $k$ iterations,
is upper triangular up to column $k$.
That is to say, the Gaussian transformations are constructed so that
all the entries in columns~1 to~$k$ and under the diagonal of the matrix $\mat A_k$ are equal to zero.
The resulting matrix $\mat A^{(n-1)}$ after $n-1$ iterations is then upper triangular,
and satisfies
\[
    \mat M_{n-1} \dotsc \mat M_1 \mat A = \mat A^{(n-1)}.
\]
We deduce that
\[
    \mat A = (\mat M_1^{-1} \dots \mat M_{n-1}^{-1}) \mat A^{(n-1)}.
\]
Since the first factor is lower triangular by~\eqref{eq:inverse_gaussian_transformation} and \cref{exercise:linear_product_of_lower_triangular},
this completes the $\mat L \mat U$ factorization of the matrix $\mat A$.
The product in the definition of the matrix $\mat L$ admits a simple expression.
\begin{lemma}
    \label{lemma:linear_inverse_product_gaussian_transformations}
    It holds that
    \begin{equation*}
        % \label{eq:product_inverse}
        \mat M_1^{-1} \dotsb \mat M_{n-1}^{-1}
        = (\mat I + \vect c^{(1)} \vect e_1^\t) \dotsb (\mat I + \vect c^{(n-1)} \vect e_{n-1}^\t)
        = \mat I + \sum_{i=1}^{n-1}  \vect c^{(i)} \vect e_i^\t.
    \end{equation*}
\end{lemma}
\begin{proof}
    Notice that, for $i < j$,
    \[
        \vect c^{(i)} \vect e_i^\t \vect c^{(j)} \vect e_j^\t
        = \vect c^{(i)} (\vect e_i^\t \vect c^{(j)}) \vect e_j^\t
        = \vect c^{(i)} 0 \vect e_j^\t = 0.
    \]
    The statement then follows easily by expanding the product.
\end{proof}
A byproduct of~\cref{lemma:linear_inverse_product_gaussian_transformations} is all the diagonal entries of the lower triangular matrix $\mat L$ are equal to 1;
the matrix $\mat L$ is \emph{unit lower triangular}.
In practice, therefore, these entries can be omitted and the matrices $\mat L$ and $\mat U$ can be stored in the same matrix.
The full expression of the matrix $\mat L$ given the Gaussian transformations is
\[
    \mat L =
    \begin{pmatrix}
        1 & \\
        c^{(1)}_2 & 1 \\
        c^{(1)}_3 & c^{(2)}_3 & 1 \\
        c^{(1)}_4 & c^{(2)}_4 & c^{(3)}_4 &  1 \\
        \vdots & \vdots & \vdots & & \ddots \\
        c^{(1)}_n & c^{(2)}_n & c^{(3)}_n & \hdots & c^{(n-1)}_n & 1 \\
    \end{pmatrix}
\]

Of course, the success of the strategy outlined above for the calculation of the $\mat L \mat U$ factorization hinges on
the existence of an appropriate Gaussian transformation at each iteration.
It is not difficult to show that,
if the $(k+1)$-th diagonal entry of the matrix $\mat A^{(k)}$ is nonzero for all~$k \in \{1, \dotsc, n-2\}$,
then the Gaussian transformation matrices exist and are uniquely defined.
\begin{lemma}
    \label{lemma:linear_expression_gaussian_transformations}
    Assume that $\mat A^{(k)}$ is an upper triangular up to columns $k$ included,
    with $k \leq n-2$.
    If $a^{(k)}_{k+1,k+1} > 0$,
    then there is a unique Gaussian transformation matrix $\mat M_{k+1}$ such that $\mat M_{k+1} \mat A^{(k)}$ is upper triangular up to column $k + 1$.
    It is given by $\mat I - \vect c^{(k+1)} \vect e_{k+1}^\t$ where
    \[
        \vect c^{(k+1)} =
        \begin{pmatrix}
            0 & 0 & \dots & 0 & \frac{a^{(k)}_{k+2,k}}{a^{(k)}_{k+1,k+1}} & \frac{a^{(k)}_{k+3,k}}{a^{(k)}_{k+1,k+1}} & \dots & \frac{a^{(k)}_{n,k}}{a^{(k)}_{k+1,k+1}}
        \end{pmatrix}^\t.
    \]
\end{lemma}
The diagonal elements $a^{(k)}_{k+1,k+1}$, where $k \in \{0, \dots, n-2 \}$, are called the pivots.
We assume for now that these are all nonzero,
and comment in \cref{sub:pivoting} on the case where a pivot is zero.
In this case,
the $\mat L \mat U$ factorization procedure is summarized as follows.

% \begin{algorithm}
% \caption{$\mat L \mat U$ factorization algorithm}%
% \label{algo:factorization_algorithm}%
\begin{algorithmic}
\State $\mat A^{(0)} \gets \mat A, \mat L \gets \mat I$
\For{$i \in \{0, \dotsc, n-2\}$}
    \State Construct $\mat M_{i+1}$ as in~\cref{lemma:linear_expression_gaussian_transformations}.
    \State $\mat A^{(i+1)} \gets \mat M_{i+1} \mat A^{(i)}, \mat L \gets \mat \mat \mat L \mat M_i^{-1}$
\EndFor
\State $\mat U \gets \mat A^{(n-1)}$.
\end{algorithmic}
% \end{algorithm}

Of course,
in practice it is not necessary to explicitly create the Gaussian transformation matrices,
or to perform full matrix multiplications.
A more realistic, but still very simplified, version of the algorithm in Julia is given below.
\begin{minted}{julia}
    # A is an invertible matrix of size n x n
    L = [i == j ? 1.0 : 0.0 for i in 1:n, j in 1:n]
    U = copy(A)
    for i in 1:n-1
        for r in i+1:n
            U[i, i] == 0 && error("Pivotal entry is zero!")
            ratio = U[r, i] / U[i, i]
            L[r, i] = ratio
            U[r, i:end] -= U[i, i:end] * ratio
        end
    end
    # L is lower triangular and U is upper triangular
\end{minted}

\subsection{Backward and forward substitution}%
\label{sub:backward_and_forward_substitution}
Once the $\mat L \mat U$ factorization has been completed,
the solution to the linear system can be obtained by first using forward, and then backward substitution,
which are just bespoke methods for solving linear systems with lower and upper triangular matrices, respectively.
Let us consider the case of a lower triangular system:
\[
    \mat L \vect y = \vect b
\]
Notice that the unknown $y_1$ may be obtained from the first line of the system.
Then, since $y_1$ is known, the value of $y_2$ can be obtained from the second line, etc.
A simple implementation of this algorithm is as follows:
\begin{minted}{julia}
    # L is unit lower triangular
    y = copy(b)
    for i in 2:n
        for j in 1:i-1
            y[i] -= L[i, j] * y[j]
        end
    end
\end{minted}

\subsection{Gaussian elimination with pivoting}%
\label{sub:pivoting}

The $\mat L \mat U$ factorization algorithm that
we presented in \cref{sub:lu_decomposition} relies on the assumption that the pivotal entries $A^{(k)}_{k+1,k+1}$ are nonzero.
In practice,
this assumption may not be satisfied,
but in this case a small modification to the algorithm,
called Gaussian elimination \emph{with pivoting},
can be employed.
In fact, pivoting is useful even if the pivots $A^{(k)}_{k+1,k+1}$ are nonzero,
as it enables to reduce the condition number of the algorithm,
i.e.\ the sensitivity of the output matrices~$\mat L$ and~$\mat U$ to perturbations of the input.
There are two types of pivoting:
partial pivoting, where only the rows are rearranged at each iteration,
and complete pivoting, where both the rows and the columns are rearranged at each iteration.

Showing rigorously why pivoting is useful requires a detailed analysis and is beyond the scope of this course.
In this section, we only discuss the mechanism of partial pivoting.
The idea of partial pivoting is to rearrange the lines at each iteration in such a manner that
the pivotal entry is as large as possible in absolute value.
One step of the procedure reads
\begin{equation}
    \label{eq:step_partial_pivoting}
    \mat A^{(k+1)} = \mat M_{k+1} \mat P_{k+1} \mat A^{(k)},
\end{equation}
where $\mat M_{k+1}$ is a Gaussian transformation matrix
and $\mat P_{k+1}$ is a row permutation matrix acting on rows $k+1$ to $n$.
The pivotal entry in~\eqref{eq:step_partial_pivoting} is defined as the entry in position $(k+1, k+1)$ of the product~$\mat P_{k+1} \mat A^{(k)}$.
% It appears in the denominator in the expression of the entries of the matrix $\mat M_{k+1}$.
\begin{definition}
    \label{definition:row_permutation_matrix}%
    Let $\sigma : \{1, \dotsc, n\} \to \{1, \dotsc, n\}$ be a permutation,\
    i.e.\ a bijection on the set~$\{1, \dotsc, n\}$.
    The row permutation matrix associated with $\sigma$ is the matrix with entries
    \[
        p_{ij} =
        \begin{cases}
            1 & \text{if $i = \sigma(j)$,} \\
            0 & \text{otherwise.}
        \end{cases}
    \]
    If $\sigma(i) = i$ for a subset of lines $S \subset \{1, \dotsc, n\}$,
    then we say that $\mat P$ acts only on rows with indices in $\{1, \dotsc, n\} \backslash S$.
\end{definition}
When a row permutation $\mat P$ left-multiplies a matrix $\mat B \in \real^{n \times n}$,
row $i$ of matrix $\mat B$ is moved to row index $\sigma(i)$ in the resulting matrix,
for all $i \in \{1, \dots, n\}$.
A permutation matrix has a single entry equal to~1 per row and per column,
and its inverse coincide with its transpose: $\mat P^{-1} = \mat P^\t$.

The resulting matrix $\mat A^{(n-1)}$ after $n-1$ steps of Gaussian elimination with partial pivoting is upper triangular and satisfies
\[
    \mat A^{(n-1)} = \mat M_{n-1} \mat P_{n-1} \dotsb \mat M_{1} \mat P_{1} \mat A
    \quad \Leftrightarrow \quad
     \mat A = (\mat M_{n-1} \mat P_{n-1} \dotsb \mat M_{1} \mat P_{1})^{-1} \mat A^{(n-1)}.
\]
The first factor in the decomposition of $\mat A$ is not necessarily lower triangular.
However, using the notation $\mat M = \mat M_{n-1} \mat P_{n-1} \dotsb \mat M_{1} \mat P_{1}$ and $\mat P = \mat P_{n-1} \dotsb \mat P_{1}$,
we have $\mat P \mat A = \mat P \mat M^{-1} \mat U = (\mat P \mat M^{-1}) \mat U$,
and the next lemma shows that the first factor on the right-hand side is lower triangular.
\begin{lemma}
    The matrix $\mat P \mat M^{-1}$ is lower triangular.
    It may be expressed as
    \[
        \mat P \mat M^{-1}
        = \mat I
        + (\mat P_{n-1} \dotsb \mat P_{2} \vect c_{1}) \vect e_{1}^\t
        + (\mat P_{n-1} \dotsb \mat P_{3} \vect c_{2}) \vect e_{2}^\t
        + \dotsb
        + (\mat P_{n-1} \vect c_{n-2}) \vect e_{n-2}^\t
        + \vect c_{n-1} \vect e_{n-1}^\t.
    \]
\end{lemma}
\begin{proof}
    Let $\mat M^{(k)} = \mat M_{k} \mat P_{k} \dotsb \mat M_{1} \mat P_{1}$ and $\mat P^{(k)} = \mat P_{k} \dotsb \mat P_{1}$.
    It is sufficient to show that
    \begin{equation}
        \label{eq:linear_lower_triangular_l}
        \mat P^{(k)} \bigl( \mat M^{(k)} \bigr)^{-1}
        = \mat I
        + (\mat P_{k} \dotsb \mat P_{2} \vect c_{1}) \vect e_{1}^\t
        + (\mat P_{k} \dotsb \mat P_{3} \vect c_{2}) \vect e_{2}^\t
        + \dotsb
        + (\mat P_{k} \vect c_{k-1}) \vect e_{n-2}^\t
        + \vect c_{k} \vect e_{k}^\t.
    \end{equation}
    for all $k \in \{1, \dotsc, n-1 \}$.
    The statement is clear for $k = 1$,
    and we assume by induction that this is true up to~$n-1$.
    Then notice that
    \begin{align*}
        \mat P^{(k)} \bigl( \mat M^{(k)} \bigr)^{-1}
        &= \mat P_{k} \left( \mat P^{(k-1)} \bigl( \mat M^{(k-1)} \bigr)^{-1} \right) \mat P_k^{-1} \mat M_k^{-1} \\
        &= \mat P_{k} \Bigl( \mat I
        + (\mat P_{k-1} \dotsb \mat P_{2} \vect c_{1}) \vect e_{1}^\t
        + \dotsb
        + (\mat P_{k-1} \vect c_{k-2}) \vect e_{n-2}^\t
        + \vect c_{k-1} \vect e_{k-1}^\t \Bigr) \mat P_k^{-1} \mat M_k^{-1} \\
        &= \Bigl( \mat I
        + (\mat P_k \mat P_{k-1} \dotsb \mat P_{2} \vect c_{1}) \vect e_{1}^\t
        + \dotsb
        + (\mat P_k \mat P_{k-1} \vect c_{k-2}) \vect e_{k-2}^\t
        + (\mat P_k \vect c_{k-1}) \vect e_{k-1}^\t \Bigr) \mat M_k^{-1}.
    \end{align*}
    In the last equality,
    we used that $\vect e_{i}^\t \mat P_k^{-1} = (\mat P_k \vect e_i)^\t = \vect e_i^\t$ for all $i \in \{1, \dotsc, n-1\}$,
    because the row permutation $\mat P_k$ does not affect lines~$1$ to~$k$.
    Using the expression $\mat M_k^{-1} = \mat I + \vect c^{(k)} \vect e_k^\t$ and expanding the product,
    we obtain~\eqref{eq:linear_lower_triangular_l}.
\end{proof}

\begin{exercise}
    [Inverse of Gaussian transformation]
    \label{exercise:inverse_gaussian_transformation}
    Prove the formula~\eqref{eq:inverse_gaussian_transformation}.
\end{exercise}

\begin{exercise}
    \label{exercise:linear_product_of_lower_triangular}
    Prove that the product of two lower triangular matrices is lower triangular.
\end{exercise}

\begin{exercise}
    Implement the backward substitution algorithm for solving $\mat U x = y$.
\end{exercise}

\textcolor{red}{WORK IN PROGRESS}
