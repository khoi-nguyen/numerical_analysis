\documentclass[11pt]{article}
\usepackage{setspace}
\onehalfspacing
\usepackage[colorlinks=true]{hyperref}
\usepackage[outputdir=build,newfloat]{minted}
\usepackage{algpseudocode}
\usepackage{algorithm}
\usepackage{mdframed}
\usepackage[margin=1.2in]{geometry}
\usepackage{amsmath,amsthm,amssymb}
\usepackage{mathtools}
\theoremstyle{definition}
\newtheorem{question}{Question}
\theoremstyle{remark}
\newtheorem*{protosolution}{Solution}
\usepackage{enumitem}
\usepackage{xcolor}
\setlist[enumerate]{font=\bfseries}
\DeclarePairedDelimiter\abs{\lvert}{\rvert}
\DeclarePairedDelimiter\norm{\lVert}{\rVert}
\DeclarePairedDelimiter\ip{\langle}{\rangle}

\DeclareMathOperator{\diag}{diag}
\DeclareMathOperator{\cond}{cond}
\DeclareMathOperator{\Span}{span}
\DeclareMathOperator{\sign}{sign}
\DeclareMathOperator{\spectrum}{spectrum}
\DeclareMathOperator*{\trace}{tr}
\DeclareMathOperator*{\argmin}{arg\,min}
\DeclareMathOperator*{\argmax}{arg\,max}

\renewcommand{\d}{\mathrm d}
\renewcommand{\t}{T}
\newcommand{\e}{\mathrm e}
\newcommand{\D}{\mathrm D}
\newcommand{\real}{\mathbf R}
\newcommand{\poly}{\mathbf P}
\newcommand{\complex}{\mathbf C}
\newcommand{\nat}{\mathbf N}
\newcommand{\integer}{\mathbf Z}
\newcommand{\floating}{\mathbf F}
\newcommand{\madd}{\mathbin{\widehat +}}
\newcommand{\mtimes}{\mathbin{\widehat *}}
\newcommand{\msub}{\mathbin{\widehat -}}
\newcommand{\vect}[1]{\mathbf{\boldsymbol{#1}}}
\newcommand{\mat}{\mathsf}
\newcommand{\placeholder}{\mathord{\color{black!33}\bullet}}%
\newcommand{\moreinfo}{\texorpdfstring{{\normalfont \color{lightred}$^{\text{\faSearchPlus}}$}}{}}
\newcommand{\laplacian}{\triangle}
\newcommand{\julia}[1]{\mintinline{julia}{#1}}


\definecolor{lightgreen}{HTML}{f1faf8}
\newenvironment{solutionframe}
{%
    \begin{mdframed}[
        leftmargin=1cm,
        skipabove=.3cm,
        linecolor=blue,
        backgroundcolor=lightgreen,
        linewidth=0pt,
        innerleftmargin=.5em,
        innerrightmargin=.5em,
        innertopmargin=.3em,
        innerbottommargin=.6em,
    ]
}
{
    \end{mdframed}
}

\newenvironment{solution}
{\pushQED{\qed}\renewcommand{\qedsymbol}{$\triangle$}
\begin{solutionframe}\small \begin{protosolution}}
{\popQED\end{protosolution}\end{solutionframe}}

\begin{document}

\title{Numerical Analysis: Midterm \\
\small{(\textbf{30 marks}, only the 3 best questions count)}}
\author{Urbain Vaes}
\maketitle

\begin{question}
    [Floating point arithmetic, 10 marks]
    True or false? (+1/0/-1)
    \begin{enumerate}
        \item
            Let $(\placeholder)_2$ denote binary representation.
            It holds that
            \(
                (0.1011)_2 + (0.0101)_2 = 1.
            \)

        \item
            Let $(\placeholder)_3$ denote base 3 representation.
            It holds that
            \(
                (1000)_3 \times (0.002)_3 = 2.
            \)

        \item
            A natural number with binary representation $(b_4 b_3 b_2 b_1 b_0)_2$ is even if and only if $b_0 = 0$.

        \item
            In Julia, \julia{Float64(.4) == Float32(.4)} evaluates to \julia{true}.

        \item
            Machine addition~$\madd$ is a commutative operation.
            More precisely, given any two double-precision floating point numbers $x \in \floating_{64}$ and $y \in \floating_{64}$,
            it holds that
            \(
                x \madd y = y \madd x.
            \)

        \item
            Let $\floating_{32}$ and $\floating_{64}$ denote respectively the sets of single and double precision floating point numbers.
            It holds that $\floating_{32} \subset \floating_{64}$.

        \item
            The machine epsilon of a floating point format is the smallest strictly positive number that can be represented exactly in the format.

        \item
            Let $\floating_{64}$ denote the set of double precision floating point numbers.
            For any $x \in \real$ such that~$x \in \floating_{64}$,
            it holds that $x + 1 \in \floating_{64}$.

        \item
            Let $a_i \in \{0, 1\}$ for $i \in \{1, 2, 3\}$.
            If $(a_1 a_2 a_3)_2$ is a multiple of 3,
            then $(a_1 a_2 a_3)_4$ is a multiple of 6.
            Here $(\placeholder)_4$ denotes base 4 representation.

        \item
            Let $f\colon \real \to \real$ denote the function that maps $x \in \real$
            to the number of double precision floating point numbers contained in the interval $[x-1, x+1]$.
            Then $f$ is a decreasing function of $x$.

        \item
            Let $n \in \nat$.
            The number of bits in the binary representation of $n$ is less than or equal to 4 times the number of digits in the decimal representation of~$n$.

        \item
            It holds that $(0.\overline{2200})_3 = (0.9)_{10}$.

        \item
            Let $p \in \nat$.
            The set
            \(
                \bigl\{ (b_0. b_1 b_2 \dots b_{p-1})_2 \colon b_i \in \{0, 1\} \bigr\}
            \)
            contains $2^{p}$ distinct real numbers.
    \end{enumerate}
    % A correct (resp.~incorrect) answer leads to +1 mark (resp. -1 mark).
\end{question}

\newpage
\begin{solution}
    The correct answers are the following:
    \begin{enumerate}
        \item True
        \item True
        \item True
        \item False, because the binary representation of $0.4$ is infinite.
        \item
            True, because
            \[
                x \madd y = {\rm fl} (x + y) = {\rm fl} (y + x) = y \madd x,
            \]
            where ${\rm fl}$ is the rounding operator.
        \item True
        \item
            False.
            The smallest number that can be represented in a format is $2^{E_{\min}-(p-1)}$,
            and the machine epsilon is $2^{-(p-1)}$.
        \item
            False, otherwise there would be infinitely many numbers in the set $\floating_{64}$.

        \item
            False. For example, $(110)_2 = 6$ and $(110)_4 = 20$.

        \item
            False since $\lim_{x \to -\infty} f(x) = 0$ and $f(0) > 0$.

        \item
            True. Indeed, let $d$ denote the number of digits in the decimal representation of~$n$.
            Then $n \leq 10^d - 1$.
            With $4d$ bits, all the numbers up to $2^{4d}- 1$ can be represented,
            and since $2^{4d} - 1 = 16^d - 1 \geq 10^d - 1$,
            the statement is true.

        \item
            True because
            \[
                (0.\overline{2200})_3 = (0.2200)_3 \Bigl(1 + 3^{-4} + (3^{-4})^2 + (3^{-4})^3 + \dotsb\Bigr)
                = \left(\frac{2}{3} + \frac{2}{9}\right)  \frac{1}{1 - 3^{-4}} = \frac{8}{9} \frac{81}{80} = \frac{9}{10}.
            \]

        \item
            True because there are $2^p$ choices for the bits,
            and distinct sets of bits correspond to distinct real numbers.
    \end{enumerate}
\end{solution}

\newpage
\begin{question}
    [Interpolation and approximation, 10 marks]
    Throughout this exercise, we assume
    that $x_0 < \dotsc < x_n$ are distinct values and
    that $u \colon \real \to \real$ is a smooth function.
    The notation $\poly(n)$ denotes the set of polynomials of degree less than or equal to $n$.

    \begin{enumerate}
        \item
            \mymarks{4}
            Are the following statements true or false? (+1/0/-1)
            \begin{itemize}
                \item
                    There exists a unique polynomial~$p \in \poly(n)$ such that
                    \begin{equation}
                        \label{eq:interpolation}
                        \forall i \in \{0, \dotsc, n\}, \qquad
                        p(x_i) = u (x_i).
                    \end{equation}

                \item
                    Assume that $p \in \poly(n)$ is such that~\eqref{eq:interpolation} is satisfied.
                    Then there is a constant~$K \in \real$ independent of~$x$ such that
                    \[
                        \forall x \in \real,
                        \qquad u(x) - p(x) = K (x - x_0) \dotsc (x - x_n).
                    \]

                \item
                    Assume that $p \in \poly(n)$ is such that~\eqref{eq:interpolation} is satisfied.
                    Then $p$ is of degree exactly~$n$.

                \item
                    If $x_0, \dotsc, x_n$ are the roots of the Chebyshev polynomial of degree~$n$,
                    then
                    \[
                        \sup_{x \in \real} \Bigl\lvert (x - x_0) \dotsc (x - x_n) \Bigr\rvert \leq \frac{\pi}{2^n}.
                    \]

                \item
                    The function $S\colon \nat \to \real$ given by
                    \[
                        S(n) = \sum_{i=1}^{n} \left(i + i^2 + i^3 + i^4\right)
                    \]
                    is a polynomial of degree 5.
                    (More precisely, there exists a polynomial of degree~5,
                    say~$q$, such that $S(n) = q(n)$ for all $n \in \nat$.)

                % \item
                %     Assume that $p \in \poly(n)$ is such that~\eqref{eq:interpolation} is satisfied.
                %     It holds that
                %     \[
                %         \sup_{x \in \real} \bigl\lvert u(x) - p(x) \bigr\rvert \leq \pi^2 /n.
                %     \]
            \end{itemize}
            \begin{solution}
                The correct answers are the following:
                \begin{itemize}
                    \item
                        True. Indeed assume that $p$ and $q$ both satisfy~\eqref{eq:interpolation}.
                        Then $p - q \in \poly(n)$ and
                        \[
                            \forall i \in \{0, \dotsc, n\}, \qquad
                            (p - q) (x_i) = 0.
                        \]
                        Therefore $p - q$ has at least $n+1$ roots which,
                        given that $p - q$ if of degree at most~$n$,
                        is possible only if $p - q = 0$.

                    \item
                        False, because if it were true,
                        then it would hold that
                        \[
                            u(x) = p(x) + K (x - x_0) \dotsc (x - x_n),
                        \]
                        implying that $u$ is a polynomial of degree~$n+1$.
                        Therefore, the equation cannot be true for a general smooth function~$u$.

                    \item
                        False.
                        The statement is not true in general since,
                        if~(for example) $u$ is  the function everywhere equal to zero,
                        then the only~$p \in \poly(n)$ that satisfies~\eqref{eq:interpolation} is $p = 0$,
                        which is not a polynomial of degree exactly~$n$.

                    \item
                        False, because the supremum on the left-hand side is equal to~$\infty$
                        as
                        \[
                            \lim_{x \to \infty} \Bigl\lvert (x - x_0) \dotsc (x - x_n) \Bigr\rvert = \infty.
                        \]

                    \item
                        True.
                \end{itemize}
            \end{solution}
    \item
        For $i \in \{0, \dotsc, n\}$,
        let $u_i = u(x_i)$,
        and let $m \leq n$ be a given natural number.
        We wish to fit the data $(x_0, u_0), \dotsc, (x_n, u_n)$ with a function $\widehat u\colon \real \to \real$ of the form
        \[
            \widehat u(x) = \alpha_0 + \alpha_1 x + \dotsc + \alpha_m x^m.
        \]
        Specifically, we wish to find coefficients $\vect \alpha = (\alpha_0, \dotsc, \alpha_m)^\t$
        such that the error
        \[
            J(\vect \alpha) := \frac{1}{2}\sum_{i=0}^{n} \abs[big]{u_i - \widehat u(x_i)}^2
        \]
        is minimized.
        Throughout this exercise,
        we use the notations
        \[
            \mat A
            \begin{pmatrix}
                1 & x_0 & \hdots & x_0^m \\
                \vdots & \vdots & & \vdots \\
                1 & x_{n} & \hdots & x_n^m
            \end{pmatrix},
            \qquad
            \vect b :=
            \begin{pmatrix}
                u_0 \\
                \vdots \\
                u_n
            \end{pmatrix}
        \]
        \begin{itemize}
            \item
            \mymarks{3}
                Show that $J(\vect \alpha)$ may be rewritten as
                \[
                    J(\vect \alpha) = \frac{1}{2} (\mat A \vect \alpha  - \vect b)^\t (\mat A \vect \alpha  - \vect b).
                \]

            \item
            \mymarks{2}
                Prove that if $\vect \alpha_* \in \real^{m+1}$ is a minimizer of $J$,
                then
                \begin{equation}
                    \label{eq:normal_equations}
                    \mat A^\t \mat A \vect \alpha_* = \mat A^\t \vect b.
                \end{equation}

            % \item
            %     \mymark
            %     Show that the matrix $\mat A^\t \mat A$ is positive definite.
            %     You can take for granted that the columns of~$\mat A$ are linearly independent.

            \item
                \mymark
                Find a solution to~\eqref{eq:normal_equations} in terms of $u_0, \dotsc, u_n$ and $n$ when $m = 0$.
                Explain.
        \end{itemize}
        \begin{solution}
            $~$
            \begin{itemize}
                \item Notice that
                    \[
                        \mat A \vect \alpha =
                        \begin{pmatrix}
                            \alpha_0 + \alpha_1 x_0 + \dotsb + \alpha_m x_0^m \\
                            \vdots \\
                            \alpha_0 + \alpha_1 x_n + \dotsb + \alpha_m x_n^m
                        \end{pmatrix}
                        =
                        \begin{pmatrix}
                            \widehat u(x_0) \\
                            \vdots \\
                            \widehat u(x_n)
                        \end{pmatrix}.
                    \]
                    Therefore
                    \[
                        \frac{1}{2} \sum_{i=1}^{n} \bigl\lvert \widehat u(x_i)  - u_i \bigr\rvert^2
                        = \frac{1}{2} \sum_{i=1}^{n} \bigl\lvert (\mat A \vect \alpha  - \vect b)_i \bigr\rvert^2
                        = \frac{1}{2} (\mat A \vect \alpha  - \vect b)^\t (\mat A \vect \alpha  - \vect b)
                    \]

                \item
                A necessary condition is that $\nabla J(\vect \alpha_*) = 0$.
                We calculate that
                \[
                    \frac{\partial}{\partial_{x_i}} \left(\vect b^\t \vect x\right)
                    = \frac{\partial}{\partial_{x_i}} \left(\sum_{j=1}^{n} b_j x_j\right) = \sum_{j=1}^{n} b_j \delta_{ij} = b_i.
                \]
                Similarly, for any matrix $\mat M \in \real^{n \times n}$,
                it holds that
                \[
                    \frac{\partial}{\partial_{x_i}} \left(\vect x^\t \mat M \vect x \right)
                    = \frac{\partial}{\partial_{x_i}} \left(\sum_{j=1}^{n}\sum_{k=1}^{n} m_{jk} x_j x_k\right)
                    =  \sum_{j=1}^{n}\sum_{k=1}^{n} m_{jk} \frac{\partial}{\partial_{x_i}} (x_j x_k).
                \]
                Applying the formula for the derivative of a product,
                we obtain
                \begin{align*}
                    \frac{\partial}{\partial_{x_i}} \left(\vect x^\t \mat M \vect x \right)
                    &= \sum_{j=1}^{n}\sum_{k=1}^{n} m_{jk} \delta_{ij} x_k + m_{jk} x_j \delta_{ik} \\
                    &= \sum_{k=1}^{n} m_{ik} x_k + \sum_{j=1}^{n} m_{ji} x_j
                    = (\mat M \vect x + \mat M^\t \vect x)_i.
                \end{align*}
                Employing these formulae,
                we calculate that (representing the gradient with a column vector)
                \[
                    \nabla_{\vect \alpha} \left(\vect b^\t \vect \alpha\right)
                    = \vect b, \qquad \nabla_{\vect \alpha} \left(\vect \alpha^\t \mat A^\t \mat A \vect \alpha\right)
                    = 2 \mat A^\t \mat A \vect \alpha.
                \]
                It is then simple to conclude.

                \item
                    In this case $\mat A^\t \mat A = n+1$ and $\alpha_*$ is a scalar.
                    The solution is given by
                    \[
                        \alpha_* = \frac{u_0 + \dotsb + u_n}{n+1},
                    \]
                    which is the average of the values $u_0, \dotsc, u_{n+1}$.
            \end{itemize}
        \end{solution}
\end{enumerate}
\end{question}

\newpage
\begin{question}
    [Numerical integration, 10 marks]
    The Gauss--Legendre quadrature formula with $n$ nodes is an approximate integration formula of the form
    \begin{equation}
        \label{gauss_legendre}
        I(u) := \int_{-1}^{1} u(x) \, \d x \approx \sum_{i=1}^{n} w_i \, u(x_i) =: \widehat I_n(u),
    \end{equation}
    which is exact when $u$ is a polynomial of degree less than or equal to $2n-1$.
    (Note that the nodes are here numbered starting from 1.)

    \begin{enumerate}
        \item
            \mymarks{5}
            Find the nodes and weights of the Gauss--Legendre rule with $n=3$ nodes.

            \begin{solution}
                A necessary and sufficient condition in order for~\eqref{gauss_legendre} to be satisfied for any polynomial~$p \in \poly(5)$ is that
                \[
                    \int_{-1}^{1} x^d \, \d x= \sum_{i=1}^{n} w_i x_i^d,
                    \qquad \text{ for all $d \in \{0, 1, 2, 3, 4, 5\}$}.
                \]
                This leads to the following system of equations
                \begin{equation*}
                    \left\{
                        \begin{aligned}
                            2 &= w_1 + w_2 + w_3, \\
                            0 &= w_1 x_1 + w_2 x_2 +  w_3 x_3, \\
                            \frac{2}{3} &= w_1 x_1^2 + w_2 x_2^2 +  w_3 x_3^2, \\
                            0 &= w_1 x_1^3 + w_2 x_2^3 +  w_3 x_3^3, \\
                            \frac{2}{5} &= w_1 x_1^4 + w_2 x_2^4 +  w_3 x_3^4, \\
                            0 &= w_1 x_1^5 + w_2 x_2^5 +  w_3 x_3^5.
                        \end{aligned}
                        \right.
                \end{equation*}
                Given the symmetry of the problem,
                it is reasonable to look for a solution of the form
                \[
                    (x_1, x_2, x_3, w_1, w_2, w_3)
                    = (-x, 0, x, w_1, w_2, w_1),
                \]
                where only 3 unknown parameters remain.
                For such a set of parameters,
                the second, fourth and sixth equations are satisfied,
                and the other three equations give
                \begin{equation*}
                    \left\{
                        \begin{aligned}
                            2 &= 2w_1 + w_2, \\
                            \frac{2}{3} &= 2 w_1 x^2, \\
                            \frac{2}{5} &= 2 w_1 x^4.
                        \end{aligned}
                        \right.
                \end{equation*}
                Dividing the third equation by the second,
                we obtain $x^2 = 3/5$ and so $x = \pm \sqrt{\frac{3}{5}}$ (both values lead to the same integration rule in the end).
                It is then simple to deduce that~$w_1 = \frac{5}{9}$ and $w_2 = \frac{8}{9}$.
                We have thus derived the formula
                \[
                    \int_{-1}^{1} u(x) \approx
                    \frac{5}{9} u \left( - \sqrt{\frac{3}{5}} \right)
                    + \frac{8}{9} u \left(0\right)
                    + \frac{5}{9} u \left( \sqrt{\frac{3}{5}} \right).
                \]
            \end{solution}

        \item
            \mymarks{2}
            Let $\{L_0, L_1, \dotsc\}$ denote orthogonal polynomials for the inner product
            \[
                \ip{f, g} := \int_{-1}^{1} f(x) g(x) \, \d x
            \]
            which, in addition, satisfy the following two conditions:
            \begin{itemize}
                \item
                    For all $i \in \nat$,
                    the polynomial $L_i$ is of degree $i$.

                \item
                    The leading coefficient of $L_i$,
                    which multiplies $x^i$,
                    is equal to 1.
            \end{itemize}
            Calculate $L_0$, $L_1$, $L_2$ and $L_3$.
            What is the connection between $L_3$ and the rule found in the first item?

            \begin{solution}
                Clearly $L_0 = 1$.
                Then $L_1 = x + a_1$ and the requirement that $\ip{L_1, L_0} = 0$ implies that $a_1 = 0$.
                We then use the ansatz $L_2 = x^2 + b_2 x + a_2$ for $L_2$.
                The requirement that $\ip{L_2, L_1}$ leads to $b_2 = 0$, and then
                \[
                    \ip{L_2, L_0} = \frac{2}{3} + 2 a_2,
                \]
                and so $L_2(x) = x^2 - \frac{1}{3}$.
                Finally, for $L_3$, we use the ansatz $L_3 = x^3 + c_3 x^2 + b_3 x + a_3$.
                We calculate
                \begin{align*}
                    \ip{L_3, 1} &= \frac{2}{3} c_3 + 2 a_3, \\
                    \ip{L_3, x} &= \frac{2}{5} + \frac{2}{3} b_3, \\
                    \ip{L_3, x^2} &= \frac{2}{5} c_3 + \frac{2}{3} a_3.
                \end{align*}
                The second equation gives $b_3 = - \frac{3}{5}$,
                and the other two equations lead to $c_3 = a_3 = 0$.
                We conclude that $L_3(x) = x^3 - \frac{3}{5}  x$.
                The roots of $L_3$ are given by $\left\{- \sqrt{\frac{3}{5}}, 0, \sqrt{\frac{3}{5}} \right\}$,
                and they coincide with the nodes of the Gauss--Legendre quadrature with 3 nodes.
            \end{solution}

        \item
            Assume that $x_1, \dotsc, x_n$ and $w_1, \dotsc, w_n$ are such that~\eqref{gauss_legendre} is satisfied for all $u \in \poly(2n-1)$.
            % denote the nodes and weights of the Gauss--Legendre quadrature with~$n$ nodes.
            \begin{itemize}

                \item
                   \mymarks{2}
                    Show that the weights are given by
                    \[
                        \forall i \in \{1, \dotsc, n\}, \qquad
                        w_i = \int_{-1}^{1} \ell_i(x) \, \d x,
                    \]
                    where $\ell_i$ is the Lagrange polynomial
                    \[
                        \ell_i(x) = \prod_{j \neq i} \frac{x - x_j}{x_i - x_j}.
                    \]

                \item
                   \mymarks{1}
                    Show that the weights are all positive: $w_i > 0$ for all $i$.
            \end{itemize}
            \begin{solution}
                Since~\eqref{gauss_legendre} holds true for all $u \in \poly(2n-1)$,
                it holds true in particular for the function~$u = \ell_i \in \poly(2n-1)$,
                which implies that
                \[
                    \int_{-1}^{1} \ell_i(x) \, \d x = \sum_{i=1}^{n} w_j \ell_i(x_j) = w_i.
                \]
                Similarly, since~\eqref{gauss_legendre} holds true also for $u \in \ell_i^2 \in \poly(2n-1)$,
                we deduce that
                \[
                    \int_{-1}^{1} \bigl(\ell_i(x)\bigr)^2 \, \d x
                    = \sum_{i=1}^{n} w_j \bigl(\ell_i(x_j)\bigr)^2 = w_i.
                \]
                Since the left-hand side is positive,
                we deduce that $w_i > 0$.
            \end{solution}

                \item
                    \mybonus{2} Prove the following error estimate:
                    if $u$ is a smooth function,
                    then
                    \[
                         \bigl\lvert I(u) - \widehat I_n(u) \bigr\rvert
                         \leq \frac{C_{2n}}{(2n)!}\int_{-1}^1 \bigl(L_n(x)\bigr)^2 \, \d x,
                         \qquad C_{2n} := \sup_{\xi \in [-1, 1]} \left\lvert u^{(2n)}(\xi) \right\rvert.
                    \]
                    \textbf{Hint}: You may find it useful to proceed as follows:
                    \begin{itemize}
                        \item
                            First show that
                            \begin{equation}
                                \label{eq:hermite_interpolation}
                                 I(u) - \widehat I_n(u)
                                = \int_{-1}^{1} u(x) - p(x) \, \d x,
                            \end{equation}
                            for \emph{any} polynomial $p \in \poly(2n-1)$ such that
                            \begin{equation}
                                \label{eq:polynomial_interp}
                                \forall i \in \{1, \dotsc, n\}, \qquad
                                p(x_i) = u(x_i).
                            \end{equation}

                        \item
                            Notice that equation~\eqref{eq:hermite_interpolation} is true in particular when $p$ is the Hermite interpolation of~$u$ at the nodes~$x_1, \dotsc, x_n$.
                            Finally, conclude by using the formula for the interpolation error proved in class:
                            if $p$ is the Hermite interpolant of $u$ at the nodes $x_1, \dotsc, x_n$,
                        then
                        \[
                            \forall x \in \real, \qquad
                            u(x) - p(x) = \frac{u^{(2n)}\bigl(\xi(x)\bigr)}{(2n)!} (x-x_1)^2 \dotsc (x - x_n)^2.
                        \]
                    \end{itemize}
                    \begin{solution}
                        Assume that $p \in \poly(2n-1)$ is such that~\eqref{eq:polynomial_interp} is satisfied.
                        Then by~\eqref{gauss_legendre} we deduce that
                        \[
                            \int_{-1}^{1} p(x) \, \d x = \sum_{i=1}^{n} w_i p(x_i) = \sum_{i=1}^{n} w_i u(x_i) = \widehat I_n(u).
                        \]
                        Consequently,
                        we obtain that
                        \[
                            I(u) - \widehat I_n(u) = \int_{-1}^1 u(x)  \, \d x - \int_{-1}^1 p(x) \, \d x = \int_{-1}^1 u(x) - p(x) \, \d x.
                        \]
                        This equation holds true in particular with $p$ being the Hermite interpolation of~$u$ at the nodes $x_1, \dotsc, x_n$.
                        Then, using the formula for the interpolation error,
                        we obtain
                        \[
                            u(x) - u(x) = \frac{u^{(2n)}\bigl(\xi(x)\bigr)}{(2n)!} (x-x_1)^2 \dotsc (x - x_n)^2
                            = \frac{u^{(2n)}\bigl(\xi(x)\bigr)}{(2n)!} \bigl(L_n(x)\bigr)^2.
                        \]
                        Indeed, as shown in class, $L_n$ is a polynomial of degree $n$ with single roots at $x_1, \dotsc, x_n$.
                        Now we conclude by noting that
                        \[
                            \bigl\lvert I(u) - \widehat I_n(u) \bigr\rvert
                            = \left\lvert \int_{-1}^{1} u(x) - p(x) \, \d x \right\rvert
                            \leq \int_{-1}^{1} \left\lvert u(x) - p(x) \right\rvert \, \d x
                            \leq \int_{-1}^{1} \frac{C_{2n}}{(2n)!} \bigl(L_n(x)\bigr)^2 \, \d x,
                        \]
                        which concludes the exercise.
                    \end{solution}
    \end{enumerate}
\end{question}

\newpage
\begin{question}
    [Vector and matrix norms, 10 marks]
    The 1-norm and the $\infty$-norm of a vector~$\vect x \in \real^n$ are defined as follows:
    \[
        \norm{\vect x}_1 = \lvert x_1 \rvert + \dotsb + \lvert x_n \rvert
        \qquad \text{ and } \qquad
        \norm{\vect x}_{\infty} = \max \Bigl\{ \lvert x_1 \rvert, \dotsc, \lvert x_n \rvert \Bigr\}.
    \]
    These norms both induce a matrix norm through the formula
    \[
        \norm{\mat A}_p
        := \sup \Bigl\{ \norm{\mat A \vect x}_p : \norm{\vect x}_p = 1 \Bigr\}.
    \]
    Prove, for $\mat A \in \real^{n \times n}$, that
    \begin{enumerate}
        \item
            \mymarks{10}
            $\norm{\mat A}_1$ is given by the maximum absolute column sum:
            \begin{equation}
                \label{eq:matrix_1_norm}
               \norm{\mat A}_1 = \max_{1 \leq j \leq n}  \sum_{i=1}^{n} \abs{a_{ij}}.
            \end{equation}

        \item
            \mybonus{2}
            $\norm{\mat A}_{\infty}$ is given by the maximum absolute row sum:
            \[
                \norm{\mat A}_{\infty} = \max_{1 \leq i \leq n}  \sum_{j=1}^{n} \abs{a_{ij}}.
            \]
    \end{enumerate}

    \textbf{Hint:} In order to prove~\eqref{eq:matrix_1_norm},
    you may find it useful to proceed as follows:
    \begin{itemize}
        \item
            Introduce $j_*$ as the index of the column with maximum absolute sum:
            \[
                j_* = \argmax_{1 \leq j \leq n}  \sum_{i=1}^{n} \abs{a_{ij}}.
            \]
        \item
            Prove the direction $\geq$ in~\eqref{eq:matrix_1_norm} by finding a vector $\vect x$ with $\norm{\vect x}_1 = 1$ such that
            \[
                \norm{\mat A \vect x}_1 =  \sum_{i=1}^{n} \abs{a_{ij_*}}.
            \]

        \item
            Prove the direction $\leq$ in~\eqref{eq:matrix_1_norm} by showing that,
            for any $\vect x \in \real^n$ with $\norm{\vect x}_1 = 1$,
            \[
               \norm{\mat A \vect x}_1 \leq  \sum_{i=1}^{n} \abs{a_{ij_*}}.
           \]
    \end{itemize}
\end{question}
\begin{solution}
    $~$
    \begin{enumerate}
        \item
            Let $\vect e_j$ denote the column vector with a 1 at entry~$j$ and zero everywhere else.
            Notice that~$\lVert \vect e_j \rVert_1 = 1$ and
            \[
                \norm{\mat A \vect e_{j_*}}_1 = \sum_{i=1}^{n} \lvert a_{ij_*} \rvert,
            \]
            and so $\norm{\mat A}_1 \geq \sum_{i=1}^{n} \lvert a_{ij_*} \rvert$.
            It remains to prove that $\norm{\mat A}_1 \leq \sum_{i=1}^{n} \lvert a_{ij_*} \rvert$.
            To this end,
            it is sufficient to show that $\norm{\mat A \vect x}_1 \leq \sum_{i=1}^{n} a_{ij_*}$ for all $\vect x \in \real^n$ with $\norm{\vect x}_1 = 1$.
            Take~$\vect x \in \real^n$ with $\norm{\vect x}_1 = 1$.
            We calculate that
            \begin{align*}
                \norm{\mat A \vect x}_1
                &= \sum_{i=1}^{n} \left\lvert \sum_{j=1}^{n} a_{ij} x_j \right\rvert
                \leq \sum_{i=1}^{n} \sum_{j=1}^{n} \lvert a_{ij} \rvert \lvert x_j \rvert  \\
                &=  \sum_{j=1}^{n} \left(\sum_{i=1}^{n} \lvert a_{ij} \rvert\right) \lvert x_j \rvert
                \leq \sum_{j=1}^{n} \left(\sum_{i=1}^{n} \lvert a_{ij_*} \rvert\right) \lvert x_j \rvert  \\
                &= \left(\sum_{i=1}^{n} \lvert a_{ij_*} \rvert\right) \sum_{j=1}^{n} \lvert x_j \rvert
                = \left(\sum_{i=1}^{n} \lvert a_{ij_*} \rvert\right) \norm{\vect x}_1 = \sum_{i=1}^{n}\lvert a_{ij_*} \rvert,
            \end{align*}
            implying that $\norm{\mat A}_1 \leq \sum_{i=1}^{n}\lvert a_{ij_*} \rvert$.

        \item
            Let~$i_*$ denote the index of a row (not necessarily unique) with maximum absolute sum,
            and let $\vect y$ be a column vector with entry~$j$ equal to $\sign(a_{i_* j})$.
            Then $\norm{\vect y}_{\infty} = 1$ and
            \[
                \norm{\mat A \vect y}_{\infty} = \sum_{j=1}^{n} \abs{a_{i_* j}},
            \]
            which implies that $\norm{\mat A}_{\infty} \geq \sum_{j=1}^{n} \abs{a_{i_* j}}$.
            It remains to prove that $\norm{\mat A}_{\infty} \leq \sum_{j=1}^{n} \abs{a_{i_* j}}$.
            To this end, take $\vect x \in \real^n$ with $\norm{\vect x}_{\infty} = 1$.
            Then for all $i \in \{1, \dotsc, n\}$,
            \begin{align*}
                \bigl\lvert (\mat A \vect x)_i \bigr\rvert
                &= \left\lvert \sum_{j=1}^{n} a_{ij} x_j \right\rvert
                \leq \sum_{j=1}^{n} \lvert a_{ij} \rvert \lvert x_j \rvert
                \leq \left( \sum_{j=1}^{n} \lvert a_{ij} \rvert \right) \max_{1\leq j \leq n} \abs{x_j} \\
                &= \left( \sum_{j=1}^{n} \lvert a_{ij} \rvert \right) \norm{\vect x}_{\infty}
                = \sum_{j=1}^{n} \lvert a_{i j} \rvert
                \leq \sum_{j=1}^{n} \lvert a_{i_* j} \rvert,
            \end{align*}
            which implies that $\norm{\mat A}_{\infty} \leq \sum_{j=1}^{n} \lvert a_{i_* j} \rvert$.
    \end{enumerate}
\end{solution}
\end{document}
