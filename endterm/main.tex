\documentclass[11pt]{article}
\usepackage{setspace}
\onehalfspacing
\usepackage[outputdir=build,newfloat]{minted}
\usepackage{algpseudocode}
\usepackage{algorithm}
\usepackage[margin=1.2in]{geometry}
\usepackage{amsmath,amsthm}
\usepackage{mathtools}
\theoremstyle{definition}
\newtheorem{question}{Question}
\DeclarePairedDelimiter\abs{\lvert}{\rvert}
\DeclarePairedDelimiter\norm{\lVert}{\rVert}
\DeclarePairedDelimiter\ip{\langle}{\rangle}

\DeclareMathOperator{\diag}{diag}
\DeclareMathOperator{\cond}{cond}
\DeclareMathOperator{\Span}{span}
\DeclareMathOperator{\sign}{sign}
\DeclareMathOperator{\spectrum}{spectrum}
\DeclareMathOperator*{\trace}{tr}
\DeclareMathOperator*{\argmin}{arg\,min}
\DeclareMathOperator*{\argmax}{arg\,max}

\renewcommand{\d}{\mathrm d}
\renewcommand{\t}{T}
\newcommand{\e}{\mathrm e}
\newcommand{\D}{\mathrm D}
\newcommand{\real}{\mathbf R}
\newcommand{\poly}{\mathbf P}
\newcommand{\complex}{\mathbf C}
\newcommand{\nat}{\mathbf N}
\newcommand{\integer}{\mathbf Z}
\newcommand{\floating}{\mathbf F}
\newcommand{\madd}{\mathbin{\widehat +}}
\newcommand{\mtimes}{\mathbin{\widehat *}}
\newcommand{\msub}{\mathbin{\widehat -}}
\newcommand{\vect}[1]{\mathbf{\boldsymbol{#1}}}
\newcommand{\mat}{\mathsf}
\newcommand{\placeholder}{\mathord{\color{black!33}\bullet}}%
\newcommand{\moreinfo}{\texorpdfstring{{\normalfont \color{lightred}$^{\text{\faSearchPlus}}$}}{}}
\newcommand{\laplacian}{\triangle}
\newcommand{\julia}[1]{\mintinline{julia}{#1}}

\begin{document}

\title{Numerical Analysis: Final exam (60 marks)}
\author{Urbain Vaes}
\maketitle

\begin{question}
    [Floating point arithmetic, \textbf{10 marks}]
    True or false? +1/-1
    \begin{enumerate}
        \item
            Let $(\placeholder)_3$ denote base 3 representation.
            It holds that
            \[
                (120)_3 + (111)_3 = (1001)_3.
            \]
        \item
            Let $(\placeholder)_2$ denote binary representation.
            It holds that
            \[
                (1000)_2 \times (0.1\overline{01})_2 = (101.\overline{01})_2.
            \]

        \item In Julia, \julia{Float64(.25) == Float32(.25)} evaluates to \julia{true}.

        \item The spacing (in absolute value) between successive double-precision (\julia{Float64}) floating point numbers is constant.

        \item
            The machine epsilon is the smallest strictly positive number that can be represented in a floating point format.

        \item
            Let $\floating_{64} \subset \real$ denote the set of double-precision floating point numbers.
            If $x \in \floating_{64}$, then~$x$ admits a finite decimal representation.

        \item
            Let $x$ be a real number. If $x \in \floating_{64}$, then $2x \in \floating_{64}$.

        \item
            The following equality holds
            \[
                (0.\overline{101})_2 = \frac{7}{3}.
            \]

        \item
            In Julia, \julia{256.0 + 2.0*eps(Float64) == 256.0} evaluates to \julia{true}.

        \item
            The set~$\floating_{64}$ of double-precision floating point numbers contains twice as many real numbers
            as the set~$\floating_{32}$ of single-precision floating point numbers.

        \item
            Let $x$ and $y$ be two numbers in $\floating_{64}$.
            The result of the machine addition $x \madd y$ is sometimes exact and sometimes not,
            depending on the values of $x$ and $y$.
    \end{enumerate}
\end{question}

\newpage
\begin{question}
    [Iterative method for linear systems, \textbf{10 marks}]
    Assume that $\mat A \in \real^{n \times n}$ is a nonsingular matrix and that $\vect b \in \real^n$.
    We wish to solve the linear system
    \begin{equation}
        \label{eq:linear_system}
        \mat A \vect x = \vect b
    \end{equation}
    using an iterative method where each iteration is of the form
    \[
        \mat M \vect x_{k+1} = \mat N \vect x_k + \vect b.
    \]
    Here $\mat A = \mat M - \mat N$ is a splitting of~$\mat A$ such that $\mat M$ is nonsingular,
    and $\vect x_k \in \real^n$ denotes the $k$-th iterate of the numerical scheme.

    \begin{enumerate}
        \item
            (3 marks)
            Let $\vect e_k := \vect x_k - \vect x_*$,
            where $\vect x_*$ is the exact solution to~\eqref{eq:linear_system}.
            Prove that
            \[
                \vect e_{k+1} = \mat M^{-1} \mat N \vect e_k.
            \]

        \item
            (3 marks)
            Let $L = \norm{\mat M^{-1} \mat N}_{\infty}$.
            Prove that
            \[
                \forall k \in \nat, \qquad
                \norm{\vect e_k}_{\infty} \leq L^k \norm{\vect e_0}_{\infty}.
            \]

        \item
            (2 marks)
            Is the condition $\norm{\mat M^{-1} \mat N}_{\infty} < 1$ necessary
            for convergence when $\vect x_0 \neq \vect x_*$?

        \item
            (2 marks)
            Assume that $\mat A$ is strictly row diagonally dominant, in the sense that
            \[
                \forall i \in \{1, \dotsc, n\}, \qquad
                \lvert a_{ii} \rvert > \sum_{j=1, j\neq i}^{n} \lvert a_{ij} \rvert.
            \]
            Show that, in this case, it holds that $\norm{\mat M^{-1} \mat N}_{\infty} < 1$ for the Jacobi method,
            i.e.\ when $\mat M$ is the diagonal of~$\mat A$.
            You may take for granted the following expression for the $\infty$-norm of a matrix $\mat X \in \real^{n \times n}$:
            \[
                \norm{\mat X}_{\infty} = \max_{1 \leq i \leq n} \sum_{j=1}^{n} \abs{x_{ij}}.
            \]

        \item
            (\textbf{Bonus +2})
            Write down a few iterations of the Jacobi method when
            \[
                \mat A =
                \begin{pmatrix}
                    1 & 2 \\
                    0 & 1
                \end{pmatrix},
                \qquad
                \vect b
                \begin{pmatrix}
                    1 \\
                    1
                \end{pmatrix},
                \qquad
                \vect x_0 =
                \begin{pmatrix}
                    0 \\
                    0
                \end{pmatrix}.
            \]
            Is the method convergent?
    \end{enumerate}
\end{question}

\newpage
\begin{question}
    [Nonlinear equations, \textbf{10 marks}]
    Assume that $\vect x_*$ is a solution to the equation
    \[
        \vect F(\vect x) = \vect x
    \]
    and that $\vect F\colon \real^n \to \real^n$ satisfies the local Lipschitz condition
    \begin{equation}
        \label{eq:local_lipschitz}
        \forall \vect x \in B_{\delta}(\vect x_*), \qquad
        \norm[big]{\vect F(\vect x) - \vect F(\vect x_*)} \leq L \norm{\vect x - \vect x_*},
    \end{equation}
    with $0 \leq L < 1$ and $\delta > 0$.
    Here $B_{\delta}(\vect x_*)$ denotes the open ball of radius $\delta$ around $\vect x_*$.
    We consider the following fixed-point iterative method for approximating~$\vect x_*$:
    \begin{equation}
        \label{eq:fixed_point}
        \vect x_{k+1} = \vect F(\vect x_k).
    \end{equation}
    Show that if $\vect x_0 \in B_{\delta}(\vect x_*)$,
    then the following statements hold:
    \begin{itemize}
        \item (3 marks) All the iterates $(\vect x_k)_{k \in \nat}$ belong to $B_{\delta}(\vect x_*)$.
        \item (3 marks) The sequence $(\vect x_k)_{k \in \nat}$ converges exponentially to $\vect x_*$.
        \item (2 marks) There is no other fixed point of $\vect F$ inside $B_{\delta}(\vect x_*)$.
    \end{itemize}
    (\textbf{3 marks})
    Explain with an example how the iterative scheme~\eqref{eq:fixed_point} can be employed for solving a nonlinear equation of the form
    \[
        \vect f(\vect x) = \vect 0.
    \]
\end{question}

% \newpage
% \begin{question}
%     Show that if $\mat A \in \real^{n \times n}$ is nonsingular,
%     then the solution to the equation $\mat A \vect x = \vect b$ belongs to the Krylov subspace
%     \[
%         \mathcal K_n(\mat A, \vect b)
%         = \Span \Bigl\{ \vect b, \mat A \vect b, \mat A^2 \vect b, \dotsc, \mat A^{n-1} \vect b \Bigr\}.
%     \]
%     You can take for granted that $\mathcal K_n(\mat A, \vect b)$ is an invariant subspace of~$\mat A$.
% \end{question}


\newpage
\begin{question}
    [Error estimate for eigenvalue problems, \textbf{10 marks}]
    Let $\norm{\placeholder}$ denote the Euclidean norm,
    and assume that~$\mat A \in \complex^{n \times n}$ is Hermitian and nonsingular.

    \begin{itemize}
        \item
            (4 marks)
            Describe a simple numerical method for calculating the eigenvalue of $\mat A$ of smallest magnitude,
            as well as the corresponding eigenvector.


        \item
            (6 marks)
            Assume that $\widehat \lambda \in \complex$ and $\widehat{\vect v} \in \complex^n$ are such that
            \begin{equation}
                \label{eq:bound}
                \norm{\mat A \widehat{\vect v} - \widehat{\lambda} \widehat{\vect v}} = \varepsilon \norm{\widehat {\vect v}}.
            \end{equation}
            Prove that there exists an eigenvalue~$\lambda$ of~$\mat A$ such that
            \[
                \abs{\lambda - \widehat \lambda} \leq \varepsilon.
            \]
    \end{itemize}

    \noindent (\textbf{Bonus +2}) Show that, in the more general case where $\mat A = \mat V \mat D \mat V^{-1}$ is diagonalizable but not necessarily Hermitian,
    equation~\eqref{eq:bound} implies the existence of an eigenvalue~$\lambda$ of~$\mat A$ with
    \[
        \abs{\widehat \lambda - \lambda} \leq \norm{\mat V} \norm{\mat V^{-1}} \varepsilon.
    \]

\end{question}

\newpage
\begin{question}
    [Interpolation error, \textbf{10 marks}]
    Let~$u$ denote the function
    \begin{align*}
        u\colon
        &[0, 2\pi] \to \real; \\
        &x \mapsto \cos(x).
    \end{align*}
    Let $p_n \colon [0, 2 \pi] \to \real$ denote the interpolating polynomial of~$u$ through the nodes
    \[
        x_i = \frac{2 \pi i}{n}, \qquad i = 0, \dotsc, n.
    \]
    \begin{itemize}
        \item
            (4 marks)
            Using a method of your choice,
            calculate $p_n$ when $n = 2$.

        \item
            (5 marks)
            Prove that
            \[
                E_n := \sup_{x \in [0, 1]} \abs{u(x) - p_n(x)}
                \leq \frac{1}{(n+1)!} \prod_{i=0}^{n} (x - x_i).
            \]
            Use the notation $e(x) = u(x) - p_n(x)$ and
            \[
                \omega_n(x) = \prod_{i=0}^{n} (x - x_i).
            \]
            
            \textbf{Hint:} You may find it useful to introduce the function
            \[
                g(t) = e(t) \omega_n(x) - e(x) \omega_n(t).
            \]
        \item (1 mark) Does the error $E_n$ tend to zero in the limit as $n \to \infty$?
    \end{itemize}

    \noindent (\textbf{Bonus +2}) Using the Gregory--Newton formula,
    find a closed expression for the sum
    \[
        S(n) = \sum_{k=1}^{n} k^2.
    \]
\end{question}

\newpage
\begin{question}
    [Numerical integration, \textbf{10 marks}]
\end{question}

\end{document}
