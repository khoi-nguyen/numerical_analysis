\documentclass[11pt]{article}
\usepackage{setspace}
\onehalfspacing
\usepackage[outputdir=build,newfloat]{minted}
\usepackage{algpseudocode}
\usepackage{algorithm}
\usepackage[margin=1.2in]{geometry}
\usepackage{amsmath,amsthm}
\usepackage{mathtools}
\usepackage{enumitem}
\usepackage[colorlinks=true]{hyperref}
\usepackage{cleveref}
\usepackage{amssymb}
\setlist[enumerate]{font=\bfseries}
\theoremstyle{definition}
\newtheorem{question}{Question}
\theoremstyle{remark}
\newtheorem*{solution}{Solution}
\DeclarePairedDelimiter\abs{\lvert}{\rvert}
\DeclarePairedDelimiter\norm{\lVert}{\rVert}
\DeclarePairedDelimiter\ip{\langle}{\rangle}

\DeclareMathOperator{\diag}{diag}
\DeclareMathOperator{\cond}{cond}
\DeclareMathOperator{\Span}{span}
\DeclareMathOperator{\sign}{sign}
\DeclareMathOperator{\spectrum}{spectrum}
\DeclareMathOperator*{\trace}{tr}
\DeclareMathOperator*{\argmin}{arg\,min}
\DeclareMathOperator*{\argmax}{arg\,max}

\renewcommand{\d}{\mathrm d}
\renewcommand{\t}{T}
\newcommand{\e}{\mathrm e}
\newcommand{\D}{\mathrm D}
\newcommand{\real}{\mathbf R}
\newcommand{\poly}{\mathbf P}
\newcommand{\complex}{\mathbf C}
\newcommand{\nat}{\mathbf N}
\newcommand{\integer}{\mathbf Z}
\newcommand{\floating}{\mathbf F}
\newcommand{\madd}{\mathbin{\widehat +}}
\newcommand{\mtimes}{\mathbin{\widehat *}}
\newcommand{\msub}{\mathbin{\widehat -}}
\newcommand{\vect}[1]{\mathbf{\boldsymbol{#1}}}
\newcommand{\mat}{\mathsf}
\newcommand{\placeholder}{\mathord{\color{black!33}\bullet}}%
\newcommand{\moreinfo}{\texorpdfstring{{\normalfont \color{lightred}$^{\text{\faSearchPlus}}$}}{}}
\newcommand{\laplacian}{\triangle}
\newcommand{\julia}[1]{\mintinline{julia}{#1}}

\begin{document}

\title{Numerical Analysis: Final exam \\
\small{(\textbf{50 marks}, only the 5 best questions count)}}
\author{Urbain Vaes}
\date{12 May 2022}
\maketitle

\begin{question}
    [Floating point arithmetic, \textbf{10 marks}]
    True or false? +1/-1
    \begin{enumerate}
        \item
            Let $(\placeholder)_3$ denote base 3 representation.
            It holds that
            \[
                (120)_3 + (111)_3 = (1001)_3.
            \]
        \item
            Let $(\placeholder)_2$ denote binary representation.
            It holds that
            \[
                (1000)_2 \times (0.1\overline{01})_2 = (101.\overline{01})_2.
            \]

        \item
            In Julia, \julia{Float64(.1) == Float32(.1)} always evaluates to \julia{true}.

        \item
            The spacing (in absolute value) between successive double-precision (\julia{Float64}) floating point numbers is constant.

        \item
            The machine epsilon is the smallest strictly positive number that can be represented in a floating point format.

        \item
            Let $\floating_{64} \subset \real$ denote the set of double-precision floating point numbers.
            If $x \in \floating_{64}$, then~$x$ admits a finite decimal representation.

        \item
            Let $x$ be a real number. If $x \in \floating_{64}$, then $2x \in \floating_{64}$.

        \item
            The following equality holds
            \[
                (0.\overline{101})_2 = \frac{7}{3}.
            \]

        \item
            In Julia, \julia{256.0 + 2.0*eps(Float64) == 256.0} evaluates to \julia{true}.

        \item
            The set~$\floating_{64}$ of double-precision floating point numbers contains twice as many real numbers
            as the set~$\floating_{32}$ of single-precision floating point numbers.

        \item
            Let $x$ and $y$ be two numbers in $\floating_{64}$.
            The result of the machine addition $x \madd y$ is sometimes exact and sometimes not,
            depending on the values of $x$ and $y$.
    \end{enumerate}
\end{question}

\newpage
\begin{question}
    [Iterative method for linear systems, \textbf{10 marks}]
    Assume that $\mat A \in \real^{n \times n}$ is a nonsingular matrix and that $\vect b \in \real^n$.
    We wish to solve the linear system
    \begin{equation}
        \label{eq:linear_system}
        \mat A \vect x = \vect b
    \end{equation}
    using an iterative method where each iteration is of the form
    \begin{equation}
        \label{eq:iterative_scheme}
        \mat M \vect x_{k+1} = \mat N \vect x_k + \vect b.
    \end{equation}
    Here $\mat A = \mat M - \mat N$ is a splitting of~$\mat A$ such that $\mat M$ is nonsingular,
    and $\vect x_k \in \real^n$ denotes the $k$-th iterate of the numerical scheme.

    \begin{enumerate}
        \item
            (\textbf{3 marks})
            Let $\vect e_k := \vect x_k - \vect x_*$,
            where $\vect x_*$ is the exact solution to~\eqref{eq:linear_system}.
            Prove that
            \[
                \vect e_{k+1} = \mat M^{-1} \mat N \vect e_k.
            \]

        \item
            (\textbf{3 marks})
            Let $L = \norm{\mat M^{-1} \mat N}_{\infty}$.
            Prove that
            \begin{equation}
                \label{eq:inequality_interpolation}
                \forall k \in \nat, \qquad
                \norm{\vect e_k}_{\infty} \leq L^k \norm{\vect e_0}_{\infty}.
            \end{equation}

        \item
            (\textbf{1 mark})
            Is the condition $\norm{\mat M^{-1} \mat N}_{\infty} < 1$ necessary
            for convergence when $\vect x_0 \neq \vect x_*$?

        \item
            (\textbf{3 marks})
            Assume that $\mat A$ is strictly row diagonally dominant, in the sense that
            \[
                \forall i \in \{1, \dotsc, n\}, \qquad
                \lvert a_{ii} \rvert > \sum_{j=1, j\neq i}^{n} \lvert a_{ij} \rvert.
            \]
            Show that, in this case, the inequality $\norm{\mat M^{-1} \mat N}_{\infty} < 1$ holds for the Jacobi method,
            i.e.\ when $\mat M$ contains just the diagonal of~$\mat A$.
            You may take for granted the following expression for the~$\infty$-norm of a matrix $\mat X \in \real^{n \times n}$:
            \[
                \norm{\mat X}_{\infty} = \max_{1 \leq i \leq n} \sum_{j=1}^{n} \abs{x_{ij}}.
            \]

        \item
            (\textbf{Bonus +1})
            Write down a few iterations of the Jacobi method when
            \[
                \mat A =
                \begin{pmatrix}
                    1 & 2 \\
                    0 & 1
                \end{pmatrix},
                \qquad
                \vect b
                \begin{pmatrix}
                    1 \\
                    1
                \end{pmatrix},
                \qquad
                \vect x_0 =
                \begin{pmatrix}
                    0 \\
                    0
                \end{pmatrix}.
            \]
            Is the method convergent?
    \end{enumerate}
\end{question}

\newpage
\begin{question}
    [Nonlinear equations, \textbf{10 marks}]
    Assume that $\vect x_* \in \real^n$ is a solution to the equation
    \[
        \vect F(\vect x) = \vect x,
    \]
    where $\vect F\colon \real^n \to \real^n$ is a smooth nonlinear function.
    We consider the following fixed-point iterative method for approximating~$\vect x_*$:
    \begin{equation}
        \label{eq:fixed_point}
        \vect x_{k+1} = \vect F(\vect x_k).
    \end{equation}
    \begin{enumerate}
        \item
            (\textbf{8 marks})
            Assume in this part that $\vect F$ satisfies the local Lipschitz condition
            \begin{equation}
                \label{eq:local_lipschitz}
                \forall \vect x \in B_{\delta}(\vect x_*), \qquad
                \norm[big]{\vect F(\vect x) - \vect F(\vect x_*)} \leq L \norm{\vect x - \vect x_*},
            \end{equation}
            with $0 \leq L < 1$ and $\delta > 0$.
            Here $B_{\delta}(\vect x_*)$ denotes the open ball of radius $\delta$ centered at~$\vect x_*$.
            Show that the following statements hold:
            \begin{itemize}
                \item (\textbf{2 marks}) There is no fixed point of $\vect F$ in $B_{\delta}(\vect x_*)$ other than $\vect x_*$.
                \item (\textbf{2 marks}) If $\vect x_0 \in B_{\delta}(\vect x_*)$, then all the iterates $(\vect x_k)_{k \in \nat}$ belong to $B_{\delta}(\vect x_*)$.
                \item (\textbf{3 marks}) If $\vect x_0 \in B_{\delta}(\vect x_*)$, then the sequence $(\vect x_k)_{k \in \nat}$ converges to~$\vect x_*$ and
                    \[
                        \forall k \in \nat, \qquad
                        \norm{\vect x_k - \vect x_*} \leq L^k \norm{\vect x_0 - \vect x_*}.
                    \]
            \end{itemize}

        \item
            (\textbf{3 marks})
            Explain with an example how the iterative scheme~\eqref{eq:fixed_point} can be employed for solving a nonlinear equation of the form
            \[
                \vect f(\vect x) = \vect 0.
            \]

        \item
            (\textbf{Bonus +1})
            Let~$\mat J_F\colon \real^n \to \real^{n\times n}$ denote the Jacobian matrix of $\vect F$.
            Show that if
            \begin{equation*}
                \forall \vect x \in B_{\delta}(\vect x_*), \qquad
                \norm[big]{\mat J_F(\vect x)} \leq L,
            \end{equation*}
            then the local Lipschitz condition~\eqref{eq:local_lipschitz} is satisfied.
    \end{enumerate}
\end{question}

\newpage

\newpage
\begin{question}
    [Error estimate for eigenvalue problem, \textbf{10 marks}]
    Let $\norm{\placeholder}$ denote the Euclidean norm,
    and assume that~$\mat A \in \real^{n \times n}$ is symmetric and nonsingular.

    \begin{enumerate}
        \item
            (\textbf{5 marks})
            Describe with words and pseudocode a simple numerical method for calculating the eigenvalue of $\mat A$ of smallest modulus,
            as well as the corresponding eigenvector.
            Assume for simplicity that this eigenvalue and the corresponding eigenvector are unique.

        \item
            (\textbf{1 mark})
            Let $\mat M \in \real^{n \times n}$ denote a nonsingular symmetric matrix.
            Prove that
            \begin{equation}
                \label{eq:intermediate_eigen}
                \forall \vect x \in \real^n, \qquad
                \norm{\mat M \vect x} \geq \norm{\mat M^{-1}}^{-1} \norm{\vect x}.
            \end{equation}
            Let $\lambda_{\min}(\mat M)$ denote the eigenvalue of~$\mat M$ of smallest modulus.
            % What is the connection between $\norm{\mat M^{-1}}$ and $\lambda_{\min}(\mat M)$?
            % \footnote{Remember that the 2-norm of a Hermitian matrix coincides with its spectral radius.}
            Deduce from~\eqref{eq:intermediate_eigen} that
            \begin{equation}
                \label{eq:bound_eigen}
                \forall \vect x \in \real^n, \qquad
                \norm{\mat M \vect x} \geq \abs{\lambda_{\min}(\mat M)} \norm{\vect x}.
            \end{equation}

        \item
            (\textbf{4 marks})
            Assume that $\widehat \lambda \in \real$ and $\widehat{\vect v} \in \real^n$ are such that
            \begin{equation}
                \label{eq:bound}
                \norm{\mat A \widehat{\vect v} - \widehat{\lambda} \widehat{\vect v}} = \varepsilon > 0,
                \qquad \norm{\widehat {\vect v}} = 1.
            \end{equation}
            Using~\eqref{eq:bound_eigen},
            prove that there exists an eigenvalue~$\lambda$ of~$\mat A$ such that
            \[
                \abs{\lambda - \widehat \lambda} \leq \varepsilon.
            \]

        \item
            (\textbf{Bonus +1}) Show that, in the more general case where $\mat A = \mat V \mat D \mat V^{-1}$ is diagonalizable but not necessarily Hermitian,
            equation~\eqref{eq:bound} implies the existence of an eigenvalue~$\lambda$ of~$\mat A$ with
            \[
                \abs{\widehat \lambda - \lambda} \leq \norm{\mat V} \norm{\mat V^{-1}} \varepsilon.
            \]
            \noindent{\textbf{Hint}:}
            Introduce $\vect r = \mat A \widehat{\vect v} - \widehat{\lambda} \widehat{\vect v}$ and rewrite
            \[
                \norm{\widehat {\vect v}} = \norm{(\mat A - \widehat \lambda \mat I)^{-1}  {\vect r}}
                = \norm{\mat V (\mat D - \widehat \lambda \mat I)^{-1}  \mat V^{-1} {\vect r}}.
            \]
    \end{enumerate}
\end{question}


\newpage
\begin{question}
    [Interpolation error, \textbf{10 marks}]
    Let~$u$ denote the function
    \begin{align*}
        u\colon
        &[0, 2\pi] \to \real; \\
        &x \mapsto \cos(x).
    \end{align*}
    Let $p_n \colon [0, 2 \pi] \to \real$ denote the interpolating polynomial of~$u$ through at the nodes
    \[
        x_i = \frac{2 \pi i}{n}, \qquad i = 0, \dotsc, n.
    \]
    \begin{enumerate}
        \item
            (\textbf{3 marks})
            Using a method of your choice,
            calculate $p_n$ for $n = 2$.

        \item
            (\textbf{6 marks})
            Let $n \in \nat_{>0}$ and $e_n(x) := u(x) - p_n(x)$.
            Prove that
            \[
                \forall x \in [0, 2\pi], \qquad
                \abs{e_n(x)}
                \leq \frac{\lvert \omega_n(x) \rvert}{(n+1)!},
            \]
            where we introduced
            \[
                \omega_n(x) := \prod_{i=0}^{n} (x - x_i).
            \]
            \textbf{Hint:} You may find it useful to introduce the function
            \[
                g(t) = e_n(t) \omega_n(x) - e_n(x) \omega_n(t).
            \]

        \item
            (\textbf{1 mark}) Does the maximum absolute error
            \[
                E_n := \sup_{x \in [0, 2\pi]} \abs{e_n(x)}
            \]
            tend to zero in the limit as $n \to \infty$?
    \end{enumerate}

    \noindent (\textbf{Bonus +1}) Using the Gregory--Newton formula,
    find a closed expression for the sum
    \[
        S(n) = \sum_{k=0}^{n} k^2.
    \]
\end{question}

\newpage
\begin{question}
    [Numerical integration, \textbf{10 marks}]
    Let $u \colon [0, 1] \to \real$ be a function we wish to integrate and
    \[
        I := \int_{0}^{1} u(x) \, \d x.
    \]
    \begin{enumerate}
        \item
        \mymarks{3}
        Consider the following integration rule:
        \begin{equation}
            \label{eq:newton_cotes}
            I\approx w_1 u(0) + w_2 u(1).
        \end{equation}
        Find the weights $w_1, w_2 \in \real$ so that this integration rule has the highest possible degree of precision.
        What is the degree of precision of the rule constructed?


        \item
        \mymarks{3}
            Let $x_i = i/n$ for $i = 0, \dotsc, n$. The composite trapezoidal rule is given by
            \begin{equation}
                \label{eq:composite_rule}
                I 
                \approx \frac{1}{2n} \bigl( u(x_0) + 2 u(x_1) + 2 u(x_2) + \dotsb + 2 u(x_{n-2}) + 2 u(x_{n-1}) + u(x_n) \bigr) =: \widehat I_n,
            \end{equation}
            Derive this rule from~\eqref{eq:newton_cotes}.

        \item
        \mymarks{3}
            Assume that $u \in C^2([0, 1])$.
            Show that, for all $n \in \nat_{>0}$,
            \begin{equation}
                \label{eq:error_bound_trapezoidal}
                \bigl\lvert I - \widehat I_n \bigr\rvert
                \leq \frac{C_2}{12 n^2},
                \qquad
                C_2 := \sup_{\xi \in [a, b]} \abs*{u''(\xi)}.
            \end{equation}

        \item
        \mymarks{1}
        In this part of the question, we assume that $u$ is a quadratic polynomial.
        It is possible to show that, in this case,
        \[
             I - \widehat I_n = - \frac{C_2}{12 n^2}.
        \]
        Explain how, given two approximations $\widehat I_n$ and $\widehat I_{2n}$ obtained with~\eqref{eq:composite_rule},
        a better approximation of the integral~$I$ can be obtained by a linear combination of the form
        \[
            \alpha_1 \widehat I_n + \alpha_2 \widehat I_{2n}.
        \]

        \item
        \mybonus{2}
        Instead of~\eqref{eq:newton_cotes},
        consider now a more general integration rule of the form
        \begin{equation}
            \label{eq:gauss_legendre}
            \int_{0}^{1} u(x) \, \d x \approx w_1 u(x_1) + w_2 u(x_2).
        \end{equation}
        Find the weights $w_1, w_2 \in \real$ and the nodes $x_1, x_2 \in [0, 1]$ so that this integration rule has the highest possible degree of precision.
        What is the degree of precision of the rule constructed?

    \end{enumerate}
\end{question}

\end{document}
