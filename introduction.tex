\chapter*{Introduction}%
Scientists across a variety of disciplines translate physical phenomena into mathematical equations.
More often than not,
these equations are too difficult to be solved analytically,
and so one has to recur to \emph{numerical simulation} in order to calculate an approximate solution.
The aim of this course is to understand the standard numerical methods for tasks ubiquitous in applied mathematics:
the solution of linear and nonlinear equations, 
the solution of eigenvalue problems,
interpolation and approximation of functions,
and numerical integration using quadrature.
Throughout the course, the \texttt{Julia} programming language will be employed to exemplify the concepts.

These lecture notes cover six important topics in numerical analysis.

\begin{itemize}
    \item
        \textbf{Floating point arithmetic.}
        In~\cref{cha:rounding_errors},
        we discuss how real numbers are represented and stored on a computer.
        We discuss in particular the~\emph{IEEE 754} standard, which is widely accepted today
        and employed in most programming languages, including \texttt{Julia}.

    \item
        \textbf{Solution of linear systems.}
        Linear systems occur in all branches of science.
        Importantly, they arise resulting from the discretization of a linear partial differential equation.
\end{itemize}
